% font-size as in "Richtlinien Semesterarbeit"
\documentclass[12pt, twoside]{extarticle}
% \usepackage{showframe}
\usepackage[utf8]{inputenc}
\usepackage[table]{xcolor}
\usepackage{lipsum}
% \usepackage{graphicx}
\usepackage[german]{babel}
% Set margin analogous to word template
\usepackage[left=2.5cm, right=2.5cm, top=2.5cm, bottom=2cm]{geometry}
% sans-serif font
%% \renewcommand{\familydefault}{\sfdefault}
% line-height as in "Richtilinien Semesterarbeit"
\renewcommand{\baselinestretch}{1.15}
% custom paragraph spacing
\setlength{\parskip}{0.25em}

\newcommand{\sectionUnnumbered}[1] {
  \section*{#1}
  \addcontentsline{toc}{section}{#1}
}

% header / footer
\usepackage{fancyhdr}

\title{Grobkonzept}
\author{Sven Tschui}
\date{October 2018}

\begin{document}


\begin{titlepage}
    {
    	\centering
    	\vspace*{2cm}
    	% \includegraphics[width=0.15\textwidth]{example-image-1x1}\par\vspace{1cm}
    	{\LARGE\bfseries\fontfamily{ppl}\selectfont Automatisierte Rechnungsverarbeitung in der Krankenversicherung mittels Deep Learning \par}
    	\vspace{1cm}
    	{\large Bachelor Thesis\par}
    	\vspace{1.5cm}
    	{Zürcher Fachhochschule\par}
    	{\bfseries\large\fontfamily{ppl}\selectfont HWZ Hochschule für Wirtschaft Zürich\par}
    	\vfill
    	{Eingereicht bei\par}
    	\vspace{0.1cm}
    	{\large Dr. Oliver Zenklusen\par}
    	\vfill
    }
    {
        \renewcommand{\arraystretch}{1.5}
        \setlength{\tabcolsep}{0pt}
        \begin{flushleft}
    	\begin{tabular}{ l@{\hspace{1.5cm}} l }
         vorgelegt von: & Sven Tschui \\
         Matrikelnummer: & 15-522-345 \\
         Studiengang: & BWI-A15 \\
         Ort, Datum & Winterthur, 2. Mai 2019 \\
        \end{tabular}
        \end{flushleft}
    }
\end{titlepage}

\pagenumbering{Roman} 

\sectionUnnumbered{Management Summary}

% TODO: Text here
\lipsum[2]

\newpage

\tableofcontents

\newpage

\sectionUnnumbered{Ehrenwörtliche Erklärung}

\vfill

Ich bestätige hiermit, dass ich
\begin{itemize}
    \item die vorliegende Thesis selbständig und ohne Benützung anderer als der angegebenen Quellen und Hilfsmittel anfertigte,
    \item die benutzten Quellen wörtlich oder inhaltlich als solche kenntlich machte,
    \item diese Arbeit in gleicher oder ähnlicher Form noch keiner Prüfungskommission vorlegte.
\end{itemize}

\vspace{2cm}


\begin{flushleft}
Winterthur, 2. Mai 2019

\vspace{1cm}

{.................................................\par}
{Sven Tschui\par}

\end{flushleft}

\vfill
\vfill

\newpage

\sectionUnnumbered{Abkürzungsverzeichnis}
{
    \definecolor{ccc}{rgb}{0.8,0.8,0.8}

    \renewcommand{\arraystretch}{1.5}

    \begin{flushleft}
    \begin{table}[ht]
    \begin{tabular}{|l|l|}
     \rowcolor{ccc}
     \hline
     \textbf{Abkürzung}&\textbf{Erklärung} \\
     \hline
     CNN & Convolutional Neuronal Network \\
     \hline
     RNN & Recurrent Neuronal Network \\
     \hline
     LSTM & Long-Short Term Memory \\
     \hline
    \end{tabular}
    \caption{Abkürzungsverzeichnis}
    \end{table}
    \end{flushleft}
}
\newpage

\pagenumbering{arabic} 

\section{Einleitung}

% TODO: Text here
\lipsum[2]

\newpage

\pagestyle{fancy}
\renewcommand{\headrulewidth}{0pt}
\renewcommand{\footrulewidth}{0pt}
\renewcommand{\sectionmark}[1]{\markboth{#1}{#1}}
\renewcommand{\subsectionmark}[1]{\markright{#1}}
\fancyhf{}
\fancyhead[RE,LO]{\nouppercase{\leftmark}}
\fancyfoot[LE,RO]{\thepage}

\section{Lorem ipsum}

% TODO: Text here
\lipsum[2]

\newpage

\subsection{Sub-section Lorem ipsum}

% TODO: Text here
\lipsum[2]

\newpage

\section{Lorem ipsum 2}

% TODO: Text here
\lipsum[2]

\section{Anhang}

\subsection{Literaturverzeichnis}

...

\newpage

\subsection{Abbildungsverzeichnis}

\listoffigures

\newpage

\subsection{Tabellenverzeichnis}

\listoftables

\newpage


\end{document}