\documentclass{article}
\usepackage[utf8]{inputenc}
\usepackage[german]{babel}

\title{Grobkonzept}
\author{sventschui }
\date{October 2018}

\begin{document}

\maketitle


\section{Ausgangslage, Forschungsproblem und -frage}

% In der Ausgangslage wird das Thema zuerst allgemein vorgestellt, dann wird auf einen bestimmten Teilaspekt des Themengebiets fokussiert. 
% Diese Fokussierung führt zum Forschungsproblem und damit zu den Erkenntnissen, die gewonnen werden sollen. Gründe werden aufgeführt, weshalb es relevant ist, das gewählte Problem zu untersuchen. Ausserdem wird der Wissensstand im Bereich des Forschungsproblems (was weiss man bereits, was noch nicht) knapp beschrieben. 
% Die Forschungsfrage schliesslich bündelt die zentralen Aspekte des Forschungsproblems als zugespitzte Frage. Die Frage sollte bereits so konkret sein, dass sie in einer Thesis untersucht werden kann.
% (Bitte diesen Text jeweils nicht löschen. Er dient als Information für die Betreuungsperson.)

\section{Zielsetzungen, inhaltliche Abgrenzung}

% In der Zielsetzung werden neben der Beantwortung der Forschungsfrage die darüber hinausgehenden Ziele benennt, die mit der Thesis verfolgt werden (was soll mit der Untersuchung erreicht werden, wer kann welchen Nutzen aus der Thesis ziehen?). 
% Indem angegeben wird, was (Forschungsfrage) warum (Zielsetzung) untersucht werden soll, kann auch definiert werden, welche Fragen, Inhalte und Ziele in der Thesis nicht verfolgt, also bewusst ausgeklammert werden.


\section{Methodische Vorgehensweise}

% Hier wird die methodische Vorgehensweise zur Beantwortung der Forschungsfrage erläutert. Die Vorgehensweise bezieht sich auf die Art der empirischen Datenerhebung (qualitativ, quantitativ oder eine Mischform) wie auch auf die geplante Auswertung der erhobenen Daten. Methoden der Datenerhebung sind beispielsweise eine Umfrage oder Interviews, Methoden der Datenanalyse sind beispielsweise statistische Tests oder eine Inhaltsanalyse. Es wird beschrieben, wie bei der Datenerhebung und -analyse vorgegangen werden soll (wer soll wie befragt werden, wie werden die Daten analysiert) und welche kritischen Aspekte in der Erhebung und Analyse zu erkennen sind
 
 
\section{Provisorisches Inhaltsverzeichnis} 

% Im provisorischen Inhaltsverzeichnis werden die thematischen Schwerpunkte der Thesis und welche wissenschaftlichen Theorien und Erklärungsansätze zur Beantwortung der Forschungsfrage herangezogen werden, definiert. Die Überschriften der Hauptkapitel der Thesis lassen die relevanten Teilaspekte des Themas und das methodische Vorgehen erkennen. Weitere Hinweise zum Aufbau der Thesis befinden sich in den Richtlinien für die Erstellung von Bachelor und Master Theses, Punkt 5.

\section{Meilensteine}

% Hier werden die wichtigsten Arbeitsschritte vom Erstellen des Grobkonzeptes bis zur Abgabe der Bachelor Thesis als Meilensteine wiedergegeben und der Betreuungsperson 3 Besprechungstermine vorgeschlagen (siehe Punkt 2 der Spezifischen Regelungen für Bachelor Theses). Die Betreuungsperson bestätigt bei der Prüfung des Grobkonzeptes die Terminvorschläge oder schlägt andere Termine vor.

%
%
% Siehe Vorlage Grobkonzept!!!
%
%
 
Hello, here is some text without a meaning.  This text should 
show what aprinted text will look like at this place.  If you 
read this text, you will get noinformation.  Really?  Is there no information?  Is there a dience betweenthis text and some 
nonsense like d

a
wd
awd

awd
a
wdgefburn"?  Kjift { not at all!...
 
\end{document}