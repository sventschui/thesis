\subsubsection{Schlussfolgerungen}

\todo[inline, color=red]{Ab hier überarbeiten

Ab hier überarbeiten

Ab hier überarbeiten

Ab hier überarbeiten

Ab hier überarbeiten

Ab hier überarbeiten}

Der Text-basierte Ansatz erzielt eine erheblich bessere Treffergenauigkeit als der Bild-basierte Ansatz. Die Analyse der Genauigkeit des Text-basierten Ansatzes zeigt, das keine Rechnungen fälschlicherweise als Optiker Rechnungen klassifiziert wurden. Mit sechs respektive fünf fälschlicherweise als Fitness und Sportverein klassifizierten Rechnungen aus dem Testset ist auch für diese beiden Klassen die Genauigkeit hoch. Dis ist sehr positiv.

Während der Fehleranalyse konnte zudem festgestellt werden, dass sich einige dieser Fehler durch die erwähnten Massnahmen sehr wahrscheinlich ausmerzen lassen.

Es kann somit gesagt werden, dass die künstliche Intelligenz erfolgreich zur Klassifizierung von Rechnungen angewendet werden kann.


\subsection{Ausblick}

Mögliches Potential