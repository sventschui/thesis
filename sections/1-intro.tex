\section{Einleitung}

Science-Fiction Kassenschlager aus Hollywood zeigen künstliche Intelligenz als Wesen mit menschlichen Zügen, die diesen oft überlegen sind. Solche Darstellungen schüren Ängste. Einige dieser Ängste, wie die Ausrottung der Menschheit, scheinen in weiter Ferne. Andere, am Arbeitsplatz durch einen Roboter abgelöst zu werden, scheinen realer als je zuvor~\autocite{Lu2018}.

Die Thematik wird nicht nur von Massenmedien aufgegriffen, sondern auch in wissenschaftlichen Artikeln behandelt. So schreibt \textcite{Tredinnick2017}, Forscher in den Gebieten der digitalen Kultur, Technologien und neuen Medien, dass das Jahr 2017 das Jahr zu sein verspricht, in welchem die künstliche Intelligenz aus Film und Fiktion in die Arbeitswelt übergeht. 

Bereits 1951 sagte Turing, wir hätten erwarten sollen, dass die Maschinen die Kontrolle übernehmen werden. Doch so weit sind wir noch nicht. Dennoch bezeichnet \textcite{Tredinnick2017} die künstliche Intelligenz als die vierte industrielle Revolution mit einschneidenden Veränderungen in unserer Wirtschaft und unserem Leben.

Doch was genau ist künstliche Intelligenz? Künstliche Intelligenz bezeichnet Technologien, mit welchen Computern beigebracht wird, flexible und rationale Entscheidungen zu treffen. Im Gegensatz zur klassischen Vorgehensweise werden dem Computer keine klaren Regeln vorgegeben, anhand welcher er operieren soll, sondern er wird mit Eingabewerten und erwarteten Ausgaben trainiert, selbst einen Lösungsweg zu finden~\autocite{Tredinnick2017}.

Die Geschichte der künstlichen Intelligenz hat früh begonnen. Weizenbaum entwickelte bereits 1966 einen Chatbot, der durch einfache Techniken in der Lage war, den Eindruck zu vermitteln, mit einem intelligenten System zu kommunizieren. Obwohl das System auf einfache Regeln zurückgriff, um eine passende und sehr offene Aussage zu formulieren, markierte es den Startschuss für die Forschung im Gebiet der Verarbeitung von natürlicher Sprache~\autocite{Tredinnick2017}. 

1997 gelang IBM mit Deep Blue der erste Sieg in Schach gegen den damaligen Weltmeister Garry Kasparow~(\cite{Campbell} in \cite{Tredinnick2017}). 2011 gelang es IBM schliesslich mit ihrem System namens Watson\footnote{IBM Watson ist ein System, welches durch Techniken aus den Gebieten des Natural Language Processing und des Machine Learning sowie durch das Mining von strukturierten und unstrukturierten Daten in der Lage ist, Fragen zu beantworten~\autocite{Tredinnick2017}.} die amerikanische Quiz-Show Jeopardy zu gewinnen. Dieser Sieg ist ein Meilenstein für die künstliche Intelligenz. Mit Watson leistete IBM Pionierarbeit bei der Sammlung und Verarbeitung von unstrukturierten sowie strukturierten Daten~\autocite{Tredinnick2017}.

Die aufgeführten Beispiele sind Ausprägungen der sogenannten schwachen oder aufgabenspezifischen künstlichen Intelligenz. Diese Art der künstlichen Intelligenz ist auf eine spezifische Aufgabe ausgerichtet und kann diese oft besser ausführen als ein Mensch. Im Gegensatz zu einem Menschen ist diese Art der künstlichen Intelligenz aber nicht fähig andere Aufgaben zu lösen~\autocite{Lu2018}.

Künstliche Intelligenz, welche mit gleicher Flexibilität und Kreativität wie Menschen auf Ihre Umwelt reagiert, wird generelle künstliche Intelligenz genannt. Im Jahre 1993 sagte Vinge, dass die technologische Singularität\footnote{Als technologische Singularität wird jener Zeitpunkt bezeichnet, zu welchem die künstliche Intelligenz die menschliche Intelligenz überholt. Ab diesem Zeitpunkt wird sich die Technologie rasend schnell selbst weiterentwickeln. Gemäss den Vertretern dieser Theorie stellt die technologische Singularität eine enorme Gefahr für die Menschheit dar~\autocite{Tredinnick2017}.}, welche durch generelle künstliche Intelligenz erreicht würde, nur noch 30 Jahre entfernt sei. Diese 30 Jahre sind bald erreicht, doch die generelle künstliche Intelligenz steht noch immer in weiter Ferne. Statt einer generellen künstliche Intelligenz näher zu kommen, wird immer mehr erkannt, wie komplex eine solche ist~\autocite{Tredinnick2017}.

Die Zeit der generellen künstlichen Intelligenz ist noch nicht gekommen, doch findet die aufgabenspezifische künstliche Intelligenz bereits heute Anwendung~\autocite{Tredinnick2017}. Im Bereich der Produktion von Saatgut gibt es bereits mehrere Studien, welche die Lösung der Problematiken der Krankheitserkennung, Saatgutqualität sowie Phänotypisierung unter Anwendung von computergestützter Bildverarbeitung mit künstlicher Intelligenz behandeln~\autocite{Patricio2018}. 

Auch um die Produktion des neuen Airbus A350 schnellstmöglich auf Hochtouren zu bringen, wurde künstliche Intelligenz angewendet. Ein System, welches von Airbus entwickelt wurde, ermöglicht, dank künstlicher Intelligenz, in 70\% aller Unterbrüche der Produktion in kürzester Zeit eine Lösung für die Wiederinbetriebnahme auszuarbeiten~\autocite{Ransbotham2017}.

Ping An Insurance Co. of China Ltd., eine der grössten Versicherungsgesellschaften von China, verwendet künstliche Intelligenz zur Automatisierung von diversen Prozessen. Die Versicherung beschäftigt 110 Data Scientists, die bereits etliche Initiativen im Bereich der künstlichen Intelligenz umgesetzt haben~\autocite{Ransbotham2017}.

Neben diesen Pionieren erwähnen \textcite{Ransbotham2017} in Ihrer Untersuchung aber auch, dass nur 14\% der Befragten denken, dass künstliche Intelligenz aktuell einen hohen Einfluss auf Ihre Angebote und Dienstleistungen hat. Jedoch denken 63\%, dass sich dies in den nächsten 5 Jahren ändern wird und die künstliche Intelligenz einen entscheidenden Wettbewerbsvorteil bieten kann~\autocite{Ransbotham2017}.

Auch im \textcite{TheEconomist2018} wird der mögliche Wettbewerbsvorteil durch die Anwendung von künstlicher Intelligenz angesprochen. Ausserhalb des Technologie-Sektors, in Branchen, welche aktuell durch den Konkurrenzkampf geprägt sind, werden grosse Firmen durch die Anwendung künstlicher Intelligenz noch grösser werden und sich zu Monopolen entwickeln.

In den letzten Monaten wurde die künstliche Intelligenz immer greifbarer. Initiativen wie die Webseite www.fast.ai, welche zum Ziel haben die Programmierung künstlicher Intelligenz der breiten Masse zugänglich zu machen, ermöglichen es, immer mehr Anwendungsfälle umzusetzen.

In dieser Arbeit wird diskutiert, ob und wieweit Anwendungen der künstlichen Intelligenz für kleinere Unternehmen mit limitiertem Budget und Ressourcen möglich sind. Die Diskussion orientiert sich an folgender Forschungsfrage:

{
    \medskip
    \setlength{\fboxsep}{1em}
    \noindent\fcolorbox{igloo-darker}{igloo}{%
        \minipage[t]{\linewidth-2\fboxsep-2\fboxrule\relax}
            \begin{flushleft}
                \centering
                Können Geschäftsprozesse in kleineren Unternehmen durch eigenentwickelte künstliche Intelligenz automatisiert werden?
            \end{flushleft}
        \endminipage}
    \medskip
}

%\subsection{Potential der künstlichen Intelligenz bei der AXA Gesundheitsvorsorge}

% In diesem Kapitel wird ein Fallbeispiel beschrieben, in welchem die Anwendung künstlicher Intelligenz einen Wettbewerbsvorteil haben könnte. Dieser Fall wird für den Arbeitgeber des Autoren, die AXA Gesundheitsvorsorge, untersucht. Einige der Aussagen in diesem Kapitel basieren auf der Berufserfahrung des Autoren.



%Aus der aktuell halb-automatisierten Indexierung von Rechnungen bei der AXA kann gesagt werden, dass für die Automatisierung auf folgende zwei Faktoren geachtet werden muss:

%\begin{itemize}
%    \item Qualität der Indexierten Daten: Fehler in der Indexierung (z.B. 1g anstelle 500mg Tabletten) führen zu Fehlern in den Abrechnungen, welche im schlimmsten Fall eine Benachteiligung des Kunden verursachen und somit das Vertrauen des Kunden beeinträchtigen.
 %   \item Manueller Aufwand: Ein hoher Anteil an manueller Arbeit verursacht hohe Kosten, ist nicht effizient, birgt viel Potential für Fehler und kann nicht schnell skaliert werden.
%\end{itemize}

% \todo[inline, color=green]{Problematik von Anfang an prägnanter darstellen? (Der aufwändige Schritt „Indexierung“ im erwähnten Prozess? Was ist das Problem? Warum braucht es Ihr Projekt?)}

%In dieser Arbeit wird diskutiert, ob die Indexierung der eingehenden Rechnungen durch die Anwendung von künstlicher Intelligenz automatisiert werden kann.

%Für die Problemstellung ist es nicht nur relevant ob sondern auch in welcher Qualität dieser Arbeitsschritt automatisiert werden kann. Die Qualität stellt ein wichtiger Erfolgsfaktor dar, da schlechte Qualität ein Image-Schaden und somit ein Wettbewerbsnachteil nach sich ziehen könnte.

%Aufgrund der geschilderten Problematik der Indexierung von Rechnungen bei der AXA entstand die Idee, diese mit neuen Technologien zu lösen. Aus dem beschriebenen Fallbeispiel und dem branchenübergreifenden Interesse an der Anwendung der künstlichen Intelligenz zur Automatisierung von Geschäftsprozessen wird für diese Arbeit folgende Forschungsfrage definiert.


% \subsection{Problematik}

% \todo[inline]{Kapitel überarbeiten...}

% Nach Wasser und Dampf zur Mechanisierung, elektrischen Maschinen zur Massenproduktion sowie Elektronik und Informationstechnologie zur Automatisierung folgt nun die vierte industrielle Revolution~\atuocite{Schwab2015}.

% Die künstliche Intelligenz ist einer der Treiber dieser Revolution und ist bereits in vielen Bereichen, wie beispielsweise bei den selbstfahrenden Fahrzeugen, anzutreffen~\autocite{Schwab2015}.

% Bei einer industriellen Revolution vorne mit dabei zu sein ist notwendig, um das überleben einer Unternehmung zu sichern. Aus diesem Grund ist es für Unternehmen wichtig, in die Forschung und Entwicklung der künstlichen Intelligenz zu investieren.

% Aber nicht nur ein solch langfristiges Ziel steht im Fokus, viel mehr kann die künstliche Intelligenz bereits jetzt Kosten in einer Unternehmung senken, indem eine Automatisierung von zuvor undenkbarem Ausmass ermöglicht wird.

% Die Automatisierung jener Aufgaben wirkt sich nicht nur kostensenkend aus, sondern kann auch einen Wettbewerbsvorteil für eine Unternehmung schaffen, indem eine Customer-experience ermöglicht wird, mit welcher eine Differenzierung gegenüber der Konkurrenz erreicht wird.

% Die Motiviation zur Bearbeitung dieser Problmatik kommt von Arbeitgeber des Autoren, der AXA Gesundheitsvorsorge, welche 2017 gegründet wurde und starkes Wachstum anstrebt. Dieses Wachstum kann nur durch einen hohen Automatisierungsgrad gestämmt werden. Daraus wurde die folgende Forschungsfrage, welche anhand einem Fallbeispiel bearbeitet werden soll, definiert:

\subsection{Zielsetzung}

Diese Arbeit hat zum Ziel, herauszufinden, ob Anwendungen der künstlichen Intelligenz in kleineren Unternehmen möglich sind. Um sich der Forschungsfrage anzunähern, wird ein Überblick über die für diese Arbeit relevanten Begriffe, Techniken und Konzepte der künstlichen Intelligenz gegeben. 

Die Forschungsfrage wird an einem konkreten Fallbeispiel, der Rechnungseinreichung bei der AXA Gesundheitsvorsorge, untersucht. Im Rahmen dieser Arbeit wird ein Prototyp entwickelt, welcher die Machbarkeit der Automatisierung der Rechnungseinreichung aufzeigt. Ein weiteres Ziel dieser Arbeit ist es, konkrete Empfehlungen an die AXA Gesundheitsvorsorge zu geben, wie im vorliegenden Fallbeispiel weiter verfahren werden soll.


% Diese Arbeit hat zum Ziel, einen Überblick über den aktuellen Stand der künstlichen Intelligenz zu schaffen und das Potential und die Limitierungen dieser zu ergründen.

% Im weiteren werden die grundlegenden Konzepte und Implementierungen der künstliche Intelligenz zusammengefasst.

% Auf dieser Grundlage wird das Hauptziel, die Anwendbarkeit der künstlichen Intelligenz zur Automatisierung von Geschäftsprozessen, verfolgt.

% Neben der Beantwortung der Forschungsfrage ist es ein weiteres wichtiges Ziel, für die AXA Gesundheitsvorsorge eine Aussage zu treffen, inwiefern sie von der künstlichen Intelligenz profitieren kann und somit Investitionen in diesem Bereich tätigen soll.

\subsection{Vorgehen}

Im Kapitel \ref{chap:automation} wird dargelegt, weshalb eine Automatisierung eines Geschäftsprozesses erstrebenswert sein kann. Neben den Gründen für eine Automatisierung werden auch die Risiken durch eine Automatisierung erläutert. Es wird dargelegt, warum künstliche Intelligenz benötigt wird, um Geschäftsprozesse zu automatisieren. Es werden ausserdem Anwendungsbeispiele künstlicher Intelligenz in der Automatisierung aufgezeigt.

Im Kapitel \ref{chap:ai} werden Themengebiete, Konzepte und Techniken im Zusammenhang mit der künstlichen Intelligenz erläutert. Ein wichtiger Bestandteil dabei ist die Funktionsweise und Anwendung von neuronalen Netzen sowie Konzepte aus dem Natural Language Processing. Die Erklärungen dieses Kapitels dienen als Grundlage für den explorativen Teil dieser Arbeit, welcher im Kapitel \ref{chap:experiment} erläutert wird.

Im Kapitel \ref{chap:experiment} wird in einem explorativen Vorgehen das Fallbeispiel der AXA Gesundheits\-vorsorge erarbeitet. Zu Beginn wird das Fallbeispiel selbst sowie der Prozess der Rechnungseinreichung bei der AXA Gesundheitsvorsorge und die damit verbundene Problemstellung erläutert. Folgend werden Anforderungen an die Automatisierung von Rechnungen von Optikern und Fitnesscentern definiert. Es werden zwei Teilaspekte der Problemstellung durch je zwei Experimente diskutiert. Der erste Teil widmet sich der Klassifizierung der Rechnungen. Im zweiten Teil werden Lösungen zur Extraktion von Informationen aus den Rechnungen vorgestellt. In beiden Teilen werden jeweils die Ergebnisse diskutiert und Fehlerquellen analysiert. 

Im Kapitel \ref{chap:summary} wird die Forschungsfrage diskutiert und es werden Schlussfolgerungen aufgrund der Ergebnisse der Experimente getroffen. Es wird ausserdem eine Empfehlung an die AXA Gesundheitsvorsorge zum weiteren Vorgehen abgegeben.

Kapitel \ref{chap:ausblick} gibt einen Ausblick zu künftigen Forschungsfeldern sowie Techniken, welche für das erarbeitete Fallbeispiel relevant sind.

Die Arbeit wird mit einer kritischen Reflexion im Kapitel \ref{chap:reflexion} abgeschlossen.

\subsection{Inhaltliche Abgrenzung}

Als Produkt dieser Arbeit entsteht ein Prototyp, der zum Zweck hat, die Forschungsfrage zu beantworten. Der Prototyp soll als Grundlage zur Entwicklung eines produktionsreifen Systems dienen, hat selbst aber keinerlei Anspruch produktionsreif zu sein.

Techniken aus dem Bereich der künstlichen Intelligenz und des Machine Learnings werden nur oberflächlich erläutert, um ein Verständnis zu gewährleisten. Eine vollumfängliche Einführung in das Themengebiet der künstlichen Intelligenz ist nicht Teil dieser Arbeit.
