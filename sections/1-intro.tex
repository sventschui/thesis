\section{Einleitung}

Science-Fiction Kassenschlager aus Hollywood zeigen künstliche Intelligenz als Wesen mit menschlichen Zügen, die diesen oft überlegen sind. Solche Darstellungen schüren Ängste. Einige dieser Ängste, wie die Ausrottung der Menschheit, scheinen in weiter ferne, andere, am Arbeitsplatz durch einen Roboter abgelöst zu werden, scheinen realer als je zuvor~\autocite{Lu2018}.

Die Thematik wird nicht nur von Massenmedien aufgegriffen\todo{Beispiele solcher Berichte}, sondern auch in Wissenschaftlichen Artikeln behandelt. So schreibt \textcite{Tredinnick2017}, Forscher in den Gebieten der digitalen Kultur, Technologien und neuen Medien, dass dieses Jahr (2017) das Jahr zu sein verspricht, in welchem die künstliche Intelligenz aus Film und Fiktion in die Arbeitswelt übergeht. 

Bereits 1951 sagte Turing\todo{cite}, dass wir hätten erwarten sollen, dass die Maschinen die Kontrolle übernehmen werden. Doch soweit sind wir noch nicht. Dennoch bezeichnet \textcite{Tredinnick2017} die künstliche Intelligenz als die vierte industrielle Revolution.

Doch was genau ist künstliche Intelligenz? Künstliche Intelligenz bezeichnet Technologien, mit welchen Computern beigebracht wird, flexible und rationale Entscheidungen zu treffen~\autocite{Tredinnick2017}.\todo{1-2 Sätz zur Erklärung von KI}

Die Geschichte der künstlichen Intelligenz hat früh begonnen. Weizenbaum\todo{Zitat Weizenbaum 1966 in Tredinnick 2017} entwickelte bereits 1966 einen Chatbot, der durch einfache Techniken in der Lage war, den Eindruck zu vermitteln, mit einem Intelligenten System zu kommunizieren. Obwohl das System auf einfache Regeln zurückgriff, um eine passende und sehr offene Aussage zu formulieren, markierte es den Startschuss für die Forschung im Gebiet der Verarbeitung von natürlicher Sprache~\autocite{Tredinnick2017}. 

1997 gelang IBM mit Deep Blue der erste Sieg in Schach gegen den damaligen Weltmeister Garry Kasparow\todo{Campbell et al., 2002 in Tredinnick2017}. 2011 gelang es IBM schliesslich mit Watson die amerikanische Quiz-Show Jeopardy zu gewinnen. Dieser Sieg ist ein Meilenstein für die künstliche Intelligenz. Mit Watson leistete IBM Pionierarbeit bei der Sammlung und Verarbeitung von unstrukturierten sowie strukturierten Daten~\autocite{Tredinnick2017}.

Die aufgeführten Beispiele sind Ausprägungen der sogenannten \enquote{schwachen} oder \enquote{aufgaben-spezfischen} künstlichen Intelligenz. Diese Art der künstlichen Intelligenz ist auf eine spezifische Aufgabe ausgerichtet und kann diese oft besser ausführen als ein Mensch. Im Gegensatz zu einem Menschen ist diese Art der künstliche Intelligenz aber nicht fähig andere Aufgaben zu lösen\autocite{Lu2018}.

Künstliche Intelligenz, welche mit gleicher Flexibilität und Kreativität wie Menschen auf Ihre Umwelt reagiert, wird \enquote{generelle} künstliche Intelligenz genannt. Im Jahre 1993 sagte Vinge\todo{Zitieren}, dass die technologische Singularität\footnote{TODO}, welche durch generelle künstliche Intelligenz erreicht würde, nur noch 30 Jahre entfernt sei. Diese 30 Jahre sind bald erreicht, doch die generelle künstliche Intelligenz steht noch immer in weiter Ferne. Statt einer generellen künstliche Intelligenz näher zu kommen, wird immer mehr erkannt, wie komplex eine solche ist~\autocite{Tredinnick2017}.

Die Zeit der generellen künstlichen Intelligenz ist noch nicht gekommen, doch findet die aufgaben-spezifische künstliche Intelligenz bereits heute Anwendung~\autocite{Tredinnick2017}. Im Bereich der Produktion von Saatgut gibt es bereits mehrere Studien, welche die Lösung der Problematiken der Krankheitserkennung, Saatgutqualität sowie der Phänotypisierung unter Anwendung von computergestützter Bildverarbeitung mit künstlicher Intelligenz diskutieren~\autocite{Patricio2018}. 

Um die Produktion des neuen Airbus A350 schnellstmöglich auf Hochtouren zu bringen wurde künstliche Intelligenz angewendet. Ein System, welches von Airbus entwickelt wurde, ermöglicht dank künstlicher Intelligenz, in 70\% aller Unterbrüche der Produktion, in kürzester Zeit eine Lösung auszuarbeiten~\autocite{Ransbotham2017}.

Auch Ping An Insurance Co. of China Ltd., eine der grössten Versicherungsgesellschaften von China, verwendet bereits künstliche Intelligenz zur Automatisierung von diversen Kundenservices. Die Versicherung beschäftigt 110 Datascientists die bereits etliche Initiativen im Bereich der künstlichen Intelligenz umgesetzt haben~\autocite{Ransbotham2017}.

Neben diesen Pionieren erwähnen \textcite{Ransbotham2017} in Ihrer Untersuchung aber auch, dass nur 14\% der Befragten denken, dass künstliche Intelligenz aktuell einen hohen Einfluss auf Ihre Angebote und Dienstleistungen haben. Jedoch denken 63\%, dass sich dies in den nächsten 5 Jahren ändern wird und die künstliche Intelligenz ein entscheidender Wettbewerbsvorteil bieten kann. Trotz des Verständnis der künstlichen Intelligenz und des Potential einen Wettbewerbsvorteil zu schaffen, wird diese noch zu wenig angewendet~\autocite{Ransbotham2017}.

Auch im \textcite{TheEconomist2018} wird der mögliche Wettbewerbsvorteil durch die Anwendung von künstlicher Intelligenz angesprochen. Auch ausserhalb des Technologie-Sektors, in Branchen, welche aktuell durch den Konkurrenzkampf geprägt sind, werden grosse Firmen durch die Anwendung künstliche Intelligenz noch grösser und entwickeln sich zu Monopolen.

\subsection{Potential der künstlichen Intelligenz bei der AXA Gesundheitsvorsorge}

In diesem Kapitel wird ein Fallbeispiel beschrieben, in welchem die Anwendung künstlicher Intelligenz einen Wettbewerbsvorteil haben könnte. Dieser Fall wird für den Arbeitgeber des Autoren, die AXA Gesundheitsvorsorge, untersucht. Einige der Aussagen in diesem Kapitel basieren auf der Berufserfahrung des Autoren.

In der Schweiz beliefen sich die Kosten für das Gesundheitswesen im Jahr 2015 auf 77.8 Milliarden Franken. Über 35\% dieser Kosten wurden durch die obligatorische Krankenversicherung gedeckt. Weitere knapp 7\% wurden von den Zusatzversicherungen übernommen. Die Krankenversicherer finanzierten also mit knapp 42\% einen beträchtlichen Teil des Gesundheitswesens in der Schweiz~\autocite{BfS2018}.

Die Kosten des Gesundheitswesen steigen stetig an, so weisen die Zahlen vom Jahr 2016 bereits Kosten von über 80 Milliarden Franken nach~\autocite{BfS2018}. Auch in den folgenden Jahren sollen die Kosten weiter steigen. \textcite{Kirchgassner2009} begründet diesen Anstieg unter anderem mit der Veränderung der Altersstruktur, dem steigenden Wohlstand sowie den neuen Möglichkeiten in der Diagnose und Behandlung durch technischen Fortschritt.

Die Kosten, welche die Krankenversicherer tragen, werden mit einem von zwei Systemen, \textit{Tiers payant} oder \textit{Tiers garant}, vergütet (vgl. Tabelle \ref{tiers}~\autocite{EDI2017}. 

\begin{wraptable}{l}{0.53\textwidth}
    \renewcommand{\arraystretch}{1.25}
    \setlength{\tabcolsep}{5pt}
    \caption{Vergütungsmodelle bei den schweizer Krankenversicherern}
    \label{tiers}
    \begin{tabular}{| p{0.15\textwidth} | p{0.32\textwidth} |}
        \hline
         Tiers payant & Kosten werden vom Leistungserbringer direkt dem Krankenversicherer in Rechnung gestellt. \\
        \hline
         Tiers garant & Kosten werden vom Leistungserbringer dem Patienten in Rechnung gestellt, welcher die Rechnung dem Krankenversicherer zur Rückvergütung weiterleitet. \\
        \hline
    \end{tabular}
\end{wraptable}

Beim System Tiers payant belastet der Leistungserbringer (bspw. Arzt oder Apotheke) die Kosten direkt dem Krankenversicherer. Dies geschieht, indem der Patient mit seiner Versichertenkarte bezahlt. Anhand dieser Versichertenkarte, welche vom Krankenversicherer ausgestellt wird, können Deckungen für den Patienten überprüft sowie die Rechnung direkt an den Krankenversicherer übermittelt werden. In diesem Fall wird die Rechnung bereits in digitaler, strukturierter Form übermittelt~\autocite{EDI2017}. 

Werden Kosten, welche über Tiers payant abgerechnet wurden, nicht vom Krankenversicherer getragen, weil beispielsweise ein Selbstbehalt vereinbart wurde, die Franchise noch nicht aufgebraucht ist oder der Patient für diese Behandlung gar nicht versichert ist, verrechnet der Krankenversicherer die Kosten dem Patienten weiter \autocite{EDI2017}.

Das System Tiers payant wird häufig in Apotheken, beim Kauf von Medikamenten mit oder ohne ärztlichem Rezept, sowie bei allen stationären Behandlungen, gemäss Art. 42 Abs. 2 KVG, verwendet \autocite{EDI2017}.

Die Verarbeitung von Rechnungen, welche über das System Tiers payant abgerechnet werden, kann der Krankenversicherer, aufgrund der digitalen, strukturierten Daten, automatisiert gestalten~\autocite{BAG2016}.

Im Fall von Tiers garant stellt der Leistungserbringer die Rechnung direkt dem Patienten aus, welcher diese dann seinem Krankenversicherer zur Rückerstattung weiterleitet. Die Rechnung kann bei allen Krankenversicherern per Post und bei den meisten auch digital, im Kundenportal oder in der App, eingereicht werden \autocite{EDI2017}.

Rechnungen, welche per Post oder digital beim Krankenversicherer zur Rückvergütung eingehen, erreichen diesen in verschiedenen Formen und unterschiedlichster Qualität. 

Während einige Rechnungen nach dem TARMED Standard für Rückforderungsbelege strukturiert sind, sind andere formlos. Die Bandbreite dieser Formlosen Rechnungen ist gross: Von handgeschriebenen Rechnungen eines örtlichen Leistungserbringer bis hin zu strukturierten Rechnungen von Fitnessketten.

Bei der Einreichung per Post kann die Qualität durch Kaffee-Flecken, Verbleichung der Belege oder sonstige Beeinträchtigungen gemindert werden, der Krankenversicherer kann aber einiges dazu beitragen die Rechnung in hoher Qualität einzulesen. So kann er beispielsweise hochauflösende Scanner und eine optimale Beleuchtung einsetzen.

Problematischer sind Rechnungen, welche von Kund/-innen digital, sprich als Foto , an den Krankenversicherer übermittelt werden. Wird ein Foto einer Rechnung über das Kundenportal eingereicht, so hat der Krankenversicherer nur noch sehr wenig Einfluss auf die Qualität der Aufnahme. Schlechte Belichtung, kleine Auflösung und abgeschnittene Rechnungen sind nur wenige der Probleme, mit welchen der Krankenversicherer zu kämpfen hat.

Egal wie und in welcher Qualität eine Rechnung einen Krankenversicherer erreicht hat, muss dieser die Rechnung in eine elektronische, strukturierte Form bringen, damit diese dann durch ein Regelwerk verarbeitet werden kann. Dieser Vorgang wird durch verschiedenste Techniken aus den Bereich der Texterkennung und der Informationsextraktion ermöglicht. Als Texterkennung oder auch Optical Character Recongition (kurz OCR) wird ein Vorgang bezeichnet, bei welchem Handschrift oder Druckbuchstaben in eine Form gebracht werden, welche von Maschinen verstanden und bearbeitet werden kann~\autocite{Xue2014}. Unter dem Begriff Informationsextraktion oder Information extraction (kurz IE) wird der Prozess verstanden, bei welchem relevante Fakten aus einem Text gewonnen werden~\autocite{Piskorski2012}.

% \todo[inline, color=green]{Wie macht das die Post? Postfinance? Dort ist das Problem vermutlich etwas einfacher aber Lösungen gibt es schon länger?}

% Viele Krankenkassen haben diese Technologien bereits im Einsatz. Auch gibt es diverse Anbieter, wie beispielsweise die Tessi document solutions (Switzerland) GmbH oder die Cent Systems AG, welche diese Technologien oder gar umfangreiche Services in diesem Bereich anbieten. Die Erfahrung mit diesen Technologien und Providern bei der AXA Ein grosser Teil der Rechnungen muss aber, aus verschiedenen Gründen, nach wie vor manuell bearbeitet werden \todo{Quelle für manuelle arbeiten}. Dies beinhaltet sowohl die Nachbearbeitung nach der elektronischen Indexierung als auch die komplett manuelle Indexierung\todo{CITE}. 
% Die CSS Krankenkasse beschäftigt beispielsweise rund 200 Personen für diese manuelle Indexierung. Bei der Assura sind es rund 500 Personen \todo{Quelle für die Anzahl personen}.

% Im Jahr 2016 wiesen die Grundversicherer einen durchschnittlichen Verwaltungskostensatz von 4.7\% aus. Dies bedeutet  Ein guter Verwaltungskostensatz konnte die CSS Kranken-Versicherung AG im Jahr 2017 ausweisen weist im Jahr Die Indexierung dieser Rechnungen ist einer der Faktoren, welche auf die hohen Verwaltungskosten der Krankenversicherer schlägt.

Die AXA, eine internationale Versicherungsgesellschaft, sieht sich, genau wie alle anderen Krankenversicherer, ebenfalls vor der Herausforderung der Indexierung von Rechnungen. Im Jahr 2017 lancierte die AXA eine Zusatzversicherung in der Gesundheitsvorsorge im Schweizer Markt. Neben der Zusatzversicherungen selbst bietet die AXA ihren Kunden einen Rechnungs-Weiterleitungs-Service an. Das bedeutet, alle Rechnungen können der AXA gesendet werden, auch wenn diese die Grundversicherung betreffen. Rechnungen beziehungsweise Rechnungspositionen, welche die Zusatzversicherung betreffen, werden von der AXA vergütet. Rechnungspositionen, welche die Grundversicherung betreffen, werden zur Vergütung an den Grundversicherer weitergeleitet \autocite{Finanzen.ch2017}.

Gemäss einem Beitrag auf \textcite{Finanzen.ch2017} ist es das Ziel der AXA, bis im Jahr 2020 insgesamt 100'000 Kunden für die Gesundheitsvorsorge zu gewinnen. Aus dem Geschäftsbericht der CSS Gruppe für das Jahr 2017 geht hervor, dass für knapp 1.7 Millionen Kunden 16 Millionen Rechnungen geprüft wurden~\autocite{CSSGruppe2018}. Werden die durchschnittlich 9.5 Rechnungen pro Kunde der CSS Gruppe auf die Zielgrösse 100'000 Kunden der AXA hochgerechnet, so muss die AXA im Jahr 2020 knapp 1 Million Rechnungen prüfen. Damit diese Menge an Rechnungen effizient verarbeitet werden kann, ist es für die AXA wichtig, den Prozess der Indexierung möglichst automatisiert zu gestalten.

Um die Verwaltungskosten der Zusatzversicherung aufzuzeigen wird wiederum der Ge\-schäfts\-be\-richt der CSS Gruppe für das Jahr 2017 herangezogen. Der Kostensatz, welcher den Anteil der Gemeinkosten am Umsatz misst, betrug für die Grundversicherung lediglich 4\%. Im Geschäftsbereich der Zusatzverischerungen liegt dieser Kostensatz allerdings viel höher, nämlich bei 21\%~\autocite{CSSGruppe2018}. Welcher Anteil an diesen Kosten nun der Prüfung respektive der Indexierung eingehender Rechnungen zuzuschreiben ist, bleibt ein Betriebsgeheimnis. Da die Rechnungen, welche die Zusatzversicherungen betreffen, aufgrund der unterschiedlichen Leistungserbringer (Alternativmedizin, Fitness, Sportverein) viel diverser sind als jene die die Grundversicherung (meist ausgestellt durch Ärzte und Spitäler nach TARMED standard) betreffen, liegt die Vermutung nahe, dass im Bereich der Zusatzversicherung ein hoher Anteil der Verwaltungskosten der Rechnungsprüfung zugeschrieben werden kann.

Die AXA profitiert aber bei einer automatisierten Indexierung von Rechnungen nicht nur von einer Kostensenkung sondern kann damit auch einen Vorteil für Ihre Kunden generieren: Die Kunden erhalten durch die automatisierte Verarbeitung der eingereichten Rechnungen viel schneller das geforderte Geld.

%Rechnungen, egal ob diese von der AXA selbst bezahlt oder an den Grundversicherer weitergeleitet werden, müssen indexiert werden um verarbeitet werden zu können. Die Indexierung wird aktuell von einem externen Provider übernommen und ist eine \todo[color=yellow]{erläutern}Blackbox. Es ist allerdings bekannt, dass ein grosser Teil der Arbeiten, nach einem automatisierten OCR Schritt, manuell gemacht werden. Die aktuelle Indexierung birgt folgende zwei Probleme:

%\todo[inline, color=igloo]{Hinweis: Ich weiss, ist eine provokative Frage. Aber ich frage mich, ob es Sinn macht, dies so zu erwähnen, weil daraus entstehen Fragen -> Warum dieser Provider, wenn nicht klar ist, was er macht? Oder warum wird nicht das Know-how akquiriert?}

%\todo[inline, color=igloo]{Frage: Warum müssen die Rechnungen indexiert werden? Bitte kurz erklären, warum dies Unternehmen machen. }

Aus der aktuell halb-automatisierten Indexierung von Rechnungen bei der AXA kann gesagt werden, dass für die Automatisierung auf folgende zwei Faktoren geachtet werden muss:

\begin{itemize}
    \item Qualität der Indexierten Daten: Fehler in der Indexierung (z.B. 1g anstelle 500mg Tabletten) führen zu Fehlern in den Abrechnungen, welche im schlimmsten Fall eine Benachteiligung des Kunden verursachen und somit das Vertrauen des Kunden beeinträchtigen.
    \item Manueller Aufwand: Ein hoher Anteil an manueller Arbeit verursacht hohe Kosten, ist nicht effizient, birgt viel Potential für Fehler und kann nicht schnell skaliert werden.
\end{itemize}

% \todo[inline, color=green]{Problematik von Anfang an prägnanter darstellen? (Der aufwändige Schritt „Indexierung“ im erwähnten Prozess? Was ist das Problem? Warum braucht es Ihr Projekt?)}

In dieser Arbeit wird diskutiert, ob die Indexierung der eingehenden Rechnungen durch die Anwendung von künstlicher Intelligenz automatisiert werden kann.

Für die Problemstellung ist es nicht nur relevant ob sondern auch in welcher Qualität dieser Arbeitsschritt automatisiert werden kann. Die Qualität stellt ein wichtiger Erfolgsfaktor dar, da schlechte Qualität ein Image-Schaden und somit ein Wettbewerbsnachteil nach sich ziehen könnte.

Aufgrund der geschilderten Problematik der Indexierung von Rechnungen bei der AXA entstand die Idee, diese mit neuen Technologien zu lösen. Aus dem beschriebenen Fallbeispiel und dem branchenübergreifenden Interesse an der Anwendung der künstlichen Intelligenz zur Automatisierung von Geschäftsprozessen wird für diese Arbeit folgende Forschungsfrage definiert.

{
    \medskip
    \setlength{\fboxsep}{1em}
    \noindent\fcolorbox{igloo-darker}{igloo}{%
        \minipage[t]{\linewidth-2\fboxsep-2\fboxrule\relax}
            \begin{flushleft}
                \centering
                Können Geschäftsprozesse durch Künstliche Intelligenz automatisiert werden?
            \end{flushleft}
        \endminipage}
    \medskip
}

Um die Beantwortung dieser Forschungsfrage zu untersützen, werden folgende Unterfragen abgeleitet:

\begin{itemize}
    \item Was wird unter künstlicher Intelligenz verstanden?
    \item Was ist die Rechnnungsindexierung und welche Rolle spielt diese für einen Krankenversicherer?
    \item Welche Ansätze aus dem Bereich der künstlichen Intelligenz können für die Rechnungsindexierung angewendet werden?
\end{itemize}

% \subsection{Problematik}

% \todo[inline]{Kapitel überarbeiten...}

% Nach Wasser und Dampf zur Mechanisierung, elektrischen Maschinen zur Massenproduktion sowie Elektronik und Informationstechnologie zur Automatisierung folgt nun die vierte industrielle Revolution~\atuocite{Schwab2015}.

% Die künstliche Intelligenz ist einer der Treiber dieser Revolution und ist bereits in vielen Bereichen, wie beispielsweise bei den selbstfahrenden Fahrzeugen, anzutreffen~\autocite{Schwab2015}.

% Bei einer industriellen Revolution vorne mit dabei zu sein ist notwendig, um das überleben einer Unternehmung zu sichern. Aus diesem Grund ist es für Unternehmen wichtig, in die Forschung und Entwicklung der künstlichen Intelligenz zu investieren.

% Aber nicht nur ein solch langfristiges Ziel steht im Fokus, viel mehr kann die künstliche Intelligenz bereits jetzt Kosten in einer Unternehmung senken, indem eine Automatisierung von zuvor undenkbarem Ausmass ermöglicht wird.

% Die Automatisierung jener Aufgaben wirkt sich nicht nur kostensenkend aus, sondern kann auch einen Wettbewerbsvorteil für eine Unternehmung schaffen, indem eine Customer-experience ermöglicht wird, mit welcher eine Differenzierung gegenüber der Konkurrenz erreicht wird.

% Die Motiviation zur Bearbeitung dieser Problmatik kommt von Arbeitgeber des Autoren, der AXA Gesundheitsvorsorge, welche 2017 gegründet wurde und starkes Wachstum anstrebt. Dieses Wachstum kann nur durch einen hohen Automatisierungsgrad gestämmt werden. Daraus wurde die folgende Forschungsfrage, welche anhand einem Fallbeispiel bearbeitet werden soll, definiert:

\subsection{Zielsetzung}

Diese Arbeit hat zum Ziel, einen Überblick über den aktuellen Stand der künstlichen Intelligenz zu schaffen und das Potential und die Limitierungen dieser zu ergründen.

Im weiteren werden die grundlegenden Konzepte und Implementierungen der künstliche Intelligenz zusammengefasst.

Auf dieser Grundlage wird das Hauptziel, die Anwendbarkeit der künstlichen Intelligenz zur Automatisierung von Geschäftsprozessen, verfolgt.

Neben der Beantwortung der Forschungsfrage ist es ein weiteres wichtiges Ziel, für die AXA Gesundheitsvorsorge eine Aussage zu treffen, inwiefern sie von der künstlichen Intelligenz profitieren kann und somit Investitionen in diesem Bereich tätigen soll.

\subsection{Inhaltliche Abgrenzung}

\todo[inline]{Abgerenzungen wenn etwas nicht oder nur oberflächlich behandelt wird}

Als Produkt dieser Arbeit entsteht ein Prototyp, der zum Zweck hat, die Forschungsfrage zu beantworten. Der Prototyp soll später als Grundlage zur Entwicklung eines produktionsreifen Systems dienen, hat selbst aber keinerlei Anspruch produktionsreif zu sein.

Techniken wie NN werden nur Oberflächlich behandelt da ein Verständnis wichtig, eine detaillierte Auseinandersetzungen den Rahmen dieser Arbeit aber überschreten würde.

\subsection{Aufbau der Arbeit}

\todo[inline]{In einigen kurzen Sätzen den Aufbau der Arbeit skizzieren}
