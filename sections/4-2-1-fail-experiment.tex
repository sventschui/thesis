\subsubsection{Zeilen-basierte Informationsextraktion}

\todo[inline]{Ausprobierte Ansätze

- NERC mit SpaCy}

% https://www.reddit.com/r/MachineLearning/comments/53ovp9/extracting_a_total_cost_from_ocr_paper_receipt/
% -> Some cool approaches discussed there
% ---> 1. OCR -> for each line -> filter for lines with total --> classifiy is total or not
% ---> 2. Simple solution: Classify each word into binary: total/not total. Dream up some features, e.g. various regex rules, adjecent words, etc. Pick word w. Highest prob.
% -----> just using adjacent words with CountVectorizer as the features, this seems to work really well
% ---> 3. Fancy solution: treat it as a neural translation task from your input sequence of words to a single word output (the total) and use RNNs.
% ---> 4. Maybe "handcode" a few features like keywords taxes (close to total, as in distance to the word taxes is close) and amount/total that are direct keywords for total

%TC:ignore
\todo[inline]{
- Inspiriert durch reddit thread: https://www.reddit.com/r/MachineLearning/comments/53ovp9/extracting\_a\_total\_cost\_from\_ocr\_paper\_receipt/

- Zeile für Zeile klassifizieren, ob es sich um den Totalbetrag handelt

-- Features könnten folgende sein: Wörterbuch, verschiedene RegExp matches (e.g. $[0-9]+$ | $[0-9]+[.,'`´ ][0-9][05]$ | $[0-9]+[.,'`´ ][0-9]{2}$)

- Zeile mit höchster Wahrscheinlichkeit ist dann das Total

- Betrag via RegExp auslesen

- Problem: mehrere Beträge auf einer Zeile
}
