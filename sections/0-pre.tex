%#############################
% Title page
%#############################
\makeTitlepage{Automatisierung von Geschäftsprozessen durch künstliche Intelligenz: Eine explorative Untersuchung des Potentials am Beispiel der Rechnungseinreichung in der Krankenversicherung}{Dr. Oliver Zenklusen}{Sven Tschui}{15-522-345}{BWI-A15}{Winterthur, 2. Mai 2019}

\blankpage

%#############################
% Abstract
%#############################
\makeAbstract{Management Summary}{
    Die künstliche Intelligenz prägt viele Hollywood Kassenschlager. Die Produktionen aus Hollywood stellen die künstliche Intelligenz oft als Wesen mit menschlichen Zügen, die diesen überlegen sind, dar. Doch bezeichnet der Begriff künstliche Intelligenz Technologien, mit welchen Computern beigebracht wird, flexible und rationale Entscheidungen zu treffen. Es werden dem Computer dabei keine klaren Regeln, sondern Eingabe- und erwartete Ausgabewerte vorgegeben. Der Computer soll anhand dieser Beispiele lernen, für neue Eingabewerte eine Antwort vorherzusagen.
    \todo{Ganzheitliche vs. Aufgabenspezifische KI}
    
    Trotz seines bereits hohen Alters ist das Gebiet der künstlichen Intelligenz noch immer von neuen Entwicklungen geprägt. Zur Zeit werden die Technologien aus diesem Gebiet meist nur von Technologie Giganten und Start-Ups verwendet. Durch Initiativen wie www.fast.ai wird das Themengebiet zugänglicher und es wird für die breiten Massen immer einfacher Systeme mit künstlicher Intelligenz zu schaffen.
    
    Aufgrund dieser Entwicklungen, setzt sich diese Arbeit zum Ziel, herauszufinden, ob Geschäftsprozesse in kleineren Unternehmen durch eigenentwickelte künstliche Intelligenz automatisert werden kann.
    \todo{Gründe \& Risiken für Automatisierung}
    
    Zur Untersuchung werden relevante Grundlagen des Themengebiet der künstlichen Intelligenz erläutert. Darauf aufbauend wird in einem explorativem Teil ein Fallbeispiel bearbeitet. Dabei wird die Automatisierung des Prozesses zur Rechnungseinreichung bei der AXA Gesundheitsvorsorge mit Hilfe von künstlicher Intelligenz analysiert. Zur Analyse zweier Aspekte der Automatisierung des Prozesses werden zwei Modelle zur Klassifizierung der Rechnungen und eines zur Informationsextraktion geschaffen respektive optimiert. Die Modelle erreichen jeweils eine gute Genauigkeit und belegen damit, dass eine Automatisierung des Prozesses möglich ist. Die Modelle müssen allerdings weiter optimiert werden, bevor diese produktiv genutzt werden könnten. Dabei darf der Aufwand nicht unterschätzt werden, denn gewisse Modelle müssen beispielsweise spezifisch für jeden Rechnungstyp trainiert werden. Der AXA Gesundheitsvorsorge wird empfohlen, in diese Richtung weiter zu investieren und einen inkrementellen Rollout eines Systems mit künstlicher Intelligenz anzustreben.
    
    Die steigende Popularität und die damit steigenden Investitionen von Unternehmen aller Grössen sowie die Ergebnisse dienen als Beweis, um die Forschungsfrage positiv zu beantworten. Trotz den hohen Investitionskosten ist es auch für kleinere Unternehmen mit nur wenig Investitionskapital ratsam in die künstliche Intelligenz zu investieren, damit sie sich einen Wettbewerbsvorteil erschaffen beziehungsweise diesen halten können.\todo{Überarbeiten...}
    
}

\newpage

%#############################
% Table of contents
%#############################
\tableofcontents

\cleardoublepage

%#############################
% Declaration of authorship
%#############################
\makeDeclarationOfAuthorship{Winterthur}{2. Mai 2019}{Sven Tschui}

\cleardoublepage

%#############################
% Preface
%#############################
%\makePreface{Vorwort}{
%    \lipsum[3]
%}

%\cleardoublepage