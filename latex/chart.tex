\documentclass{hwz}
\usepackage{multirow}
\usepackage{hhline}
\usepackage{color}
\begin{document}

{
    \definecolor{cBluesPlus10}{HTML}{7ecbfb}
    \definecolor{cBlues}{HTML}{0788D9}
    \definecolor{cBluesPlus5}{HTML}{3CB0F9}
    \arrayrulecolor{white}
    \setlength{\tabcolsep}{8pt} % Default value: 6pt
    \renewcommand{\arraystretch}{1.15}
    \begin{figure}[h]
    \footnotesize
    \centering
        \captionsetup{width=.9\linewidth}
        \caption[Wettbewerbsstrategien nach Porter]{Wettbewerbsstrategien nach Porter, systematisiert nach strategischem Vorteil und strategischem Zielobjekt.}
        \label{porter_wettbewerb}
        \begin{tabular}{|
            >{\columncolor{cBluesPlus10}}c |
            >{\columncolor{cBluesPlus5}}l |
            >{\columncolor{cBlues}}l |
            >{\columncolor{cBlues}}l }
            \hline
            \multicolumn{2}{|l|}{\cellcolor{cBluesPlus10}{\color[HTML]{333333} }} &
                \multicolumn{2}{c|}{\cellcolor{cBluesPlus10}{\color[HTML]{333333} \textbf{\begin{tabular}[c]{@{}c@{}}Strategischer Vorteil\\ (Leistung oder Kosten)\end{tabular}}}} \\ \hline
                
            {} &
                {\color[HTML]{333333} \begin{tabular}[c]{@{}l@{}}Branchenweit\\ (Gesamtmarktabdeckung)\end{tabular}} &
                {\color[HTML]{FFFFFF} \begin{tabular}[c]{@{}l@{}}Differenzierung\\ (Qualitätsführerschaft)\end{tabular}} & 
                {\color[HTML]{FFFFFF} Kostenführerschaft}        \\ \cline{2-4} 
                
            \multirow{-2}{*}{
                \cellcolor{cBluesPlus10}{\color[HTML]{333333} \textbf{\begin{tabular}[c]{@{}c@{}}Strategisches\\ Zielobjekt\end{tabular}}}
            } & 
                {\color[HTML]{333333} \begin{tabular}[c]{@{}l@{}}Beschränkung auf Segment\\ (Teilmarktabdeckung)\end{tabular}} &
                \multicolumn{2}{l|}{\cellcolor{cBlues}{\color[HTML]{FFFFFF} Konzentration auf Nischen}} \\ \hline
        \end{tabular}
        \caption*{Quelle: \textcite{Capaul2010}}
    \end{figure}
}
\end{document}