\subsubsection{Schlussfolgerungen}

Der Text-basierte Ansatz erzielt eine erheblich bessere Trefferquote als der Bild-basierte Ansatz. Die Analyse der Vorhersagen des Text-basierten Ansatzes zeigt, dass nur wenige Rechnungen fälschlicherweise als Optiker Rechnungen klassifiziert wurden. Mit sechs respektive fünf fälschlicherweise als Fitness und Sportverein klassifizierten Rechnungen aus dem Testset, ist auch für diese beiden Klassen die Genauigkeit hoch. 


Während der Fehleranalyse konnte zudem festgestellt werden, dass sich einige dieser Fehler durch die erwähnten Massnahmen sehr wahrscheinlich beheben lassen.

Werden die erwähnten Optimierungsmassnahmen getroffen, so liefert die Text-basierte Klassifizierung eine gute Grundlage zur automatisierten Verarbeitung von Rechnungen. Bevor allerdings eine abschliessende Aussage zur Anwendung von künstlicher Intelligenz im Rechnungseinreichungsprozess getroffen werden kann, muss der Aspekt der Informationsextraktion analysiert werden.
