% font-size as in "Richtlinien Semesterarbeit"
\documentclass[12pt, twoside, table]{extarticle}
% TODOs
\usepackage[color=lightgray]{todonotes}
% colored tables
\usepackage{colortbl}
\usepackage{xcolor}
% Set margin analogous to word template
\usepackage[left=2.5cm, right=2.5cm, top=2.5cm, bottom=2cm]{geometry}
% line-height as in "Richtilinien Semesterarbeit"
\renewcommand{\baselinestretch}{1.15}
% custom paragraph spacing
\setlength{\parskip}{0.25em}
% citation style
\usepackage{apacite}
\bibliographystyle{apacite}
% plumbing...
\usepackage[utf8]{inputenc}
\usepackage[german]{babel}
% unnumbered sections 
\newcommand{\sectionUnnumbered}[1] {
  \section*{#1}
  \addcontentsline{toc}{section}{#1}
}
% custom list spacings
\usepackage{enumitem}
% custom colors
\definecolor{ccc}{rgb}{0.8,0.8,0.8}
\definecolor{igloo}{rgb}{0.7,0.81,0.93}
\definecolor{igloo-darker}{rgb}{0.2,0.3,0.43}

\title{Grobkonzept}
\author{Sven Tschui}
\date{October 2018}

\begin{document}

\begin{titlepage}
    {
    	\centering
    	
    	\vspace*{2cm}
    	% \includegraphics[width=0.15\textwidth]{example-image-1x1}\par\vspace{1cm}
    	{\Large\fontfamily{ppl}\selectfont Grobkonzept zur Bachelor Thesis}
    	
    	{\LARGE\bfseries\fontfamily{ppl}\selectfont Automatisierung von Kundenservices durch künstliche Intelligenz am Beispiel der Rückforderungseinreichung in der Krankenversicherung \par}
    	
    	\vspace{3cm}
    	
    	{Zürcher Fachhochschule\par}
    	
    	{\bfseries\large\fontfamily{ppl}\selectfont HWZ Hochschule für Wirtschaft Zürich\par}
    	
    	\vfill
    }
    {
        \renewcommand{\arraystretch}{1.5}
        \setlength{\tabcolsep}{0pt}
        \begin{flushleft}
    	\begin{tabular}{ l@{\hspace{1.5cm}} l }
         Student: & Sven Tschui \\
         Studiengruppe: & BWI-A15 \\
         Betreuungsperson: & Dr. Oliver Zenklusen \\
         Datum & 1. Dezember 2018 \\
        \end{tabular}
        \end{flushleft}
    }
\end{titlepage}

\newpage
\tableofcontents

\section{Ausgangslage, Forschungsproblem und -frage}

\todo[inline]{
In der Ausgangslage wird das Thema zuerst allgemein vorgestellt, dann wird auf einen bestimmten Teilaspekt des Themengebiets fokussiert. 

Diese Fokussierung führt zum Forschungsproblem und damit zu den Erkenntnissen, die gewonnen werden sollen. Gründe werden aufgeführt, weshalb es relevant ist, das gewählte Problem zu untersuchen. Ausserdem wird der Wissensstand im Bereich des Forschungsproblems (was weiss man bereits, was noch nicht) knapp beschrieben. 

Die Forschungsfrage schliesslich bündelt die zentralen Aspekte des Forschungsproblems als zugespitzte Frage. Die Frage sollte bereits so konkret sein, dass sie in einer Thesis untersucht werden kann.

\textit{(Bitte diesen Text jeweils nicht löschen. Er dient als Information für die Betreuungsperson.)}
}

% Schneller und guter Kundenservice ist wichtig um Kunden halten zu können, dies ist längst bekannt \todo{CITE}. Auch die Auswirkungen einer hohen Kundenzufriedenheit auf die Kundenloyalität werden als sehr stark anerkannt \todo{CITE}. Aus diesen Gründen investieren Unternehmen immer stärker in diese Services und versuchen sich damit von der Konkurrenz zu differenzieren \todo{CITE}. Diese Services bereitzustellen kann für ein Unternehmen sehr kostenintensiv sein \todo{CITE}. \todo[inline, color=orange]{ Fakten und Zahlen würden diese Aussage nicht nur untermauern sondern auch spannender machen} Kundenservices sollen heutzutage mit geeigneter Soft- und Hardware automatisiert werden, um die Kosten so gering wie möglich zu halten \todo{CITE}. Die Automatisierung hat ausserdem den Effekt, dass Services schneller abgewickelt werden können, was im Zeitalter der ungeduldigen Kunden ein wichtiger Differenzierungsfaktor sein kann \todo{CITE}. Die Automatisierung durch klassische Software stösst aber teilweise \todo{konkrekt sagen, wann diese Grenzen auftreten} an ihre Grenzen. Die klassische, strukturierte Programmierweise bildet Logik ab, welche durch Ursache-Wirkung klar definierbar ist: Wenn eine Bedingung eintrifft, wird etwas ausgeführt \todo{QUOTE}. Ein Abwägen von Fall zu Fall, wie dies ein Angestellter im Kundenservice machen könnte, ist mit solcher Software nicht möglich. Eine Lösung für diese Limitierung bietet die künstliche Intelligenz: Der Computer wird nicht mehr angewiesen was zu tun ist, sondern was das Ziel ist. Der Computer wird dann auf die erreichung dieses Ziels trainiert \todo{QUOTE}.

% In dieser Arbeit werden die Möglichkeiten diskutiert, welche die künstliche Intelligenz für die Automatisierung von Kundenservices bietet. Im ersten Teil werden die grundlegenden Elemente der Künstlichen Intelligenz erläutert und Herausforderungen in der Automatisierung der Kundenservices aufgezeigt. \todo[inline]{weiterer Inhalt, e.g. theoretische erfahrung, Praxisbeispiele} Der zweite Teil widmet sich einem Fallbeispiel eines Kundenservices, welcher in einem Prototypen durch den Einsatz von Künstlicher Intelligenz automatisiert werden soll. Der Erfolg des Experiments wird anhand zuvor definierten Erfolgskriterien diskutiert.

% \subsection{Fallbeispiel}

In der Schweiz beliefen sich die Kosten für das Gesundheitswesen im Jahr 2015 auf 77.8 Milliarden Franken. Über 35\% dieser Kosten wurden durch die obligatorische Krankenversicherung gedeckt. Weitere knapp 7\% wurden durch Zusatzversicherungen übernommen. Die Krankenversicherer finanzierten also mit knapp 42\% einen beträchtlichen Teil des Gesundheitswesens in der Schweiz \cite{BundesamtfurStatistik2018Finanzierung, BundesamtfurStatistik2017KostenDaten}.

Die Kosten des Gesundheitswesen steigen stetig an, so weisen die Zahlen vom Jahr 2016 bereits Kosten von über 80 Milliarden Franken nach. Auch in den folgenden Jahren sollen die Kosten weiter steigen \todo{CITE for this trend} \cite{BundesamtfurStatistik2018Finanzierung}.

Die Kosten, welche die Krankenversicherer tragen, werden mit einem von zwei Systemen, Tiers payant oder Tiers garant, vergütet \cite{EidgenossischesDepartementdesInnern2017FaktenblattVergutungssysteme}. 

Beim System Tier payant belastet der Leistungserbringenr (bspw. Arzt oder Apotheke) die Kosten direkt dem Krankenversicherer. Dies geschieht, in dem der Patient mit seiner Kranken\-versicherungs-Karte bezahlt. Anhand dieser Karte, welche vom Krankenversicherer ausgestellt wird, können Deckungen für den Patienten überprüft sowie die Rechnung direkt an den Versicherer übermittelt werden. In diesem Fall wird die Rechnung bereits in digitaler, strukturierter Form übermittelt und der Krankenversicherer kann mit einem entsprechenden Regelwerk die Rechnung automatisch verarbeiten. 

Werden Kosten, welche über Tier payant abgerechnet wurden, nicht vom Krankenversicherer getragen, weil beispielsweise ein Selbstbehalt vereinbart wurde, die Franchise noch nicht aufgebraucht ist oder der Patient für diese Behandlung gar nicht versichert ist, verrechnet der Versicherer die Kosten dem Patienten weiter \cite{EidgenossischesDepartementdesInnern2017FaktenblattVergutungssysteme}.

Das System Tier payant wird häufig in Apotheken, beim Kauf von Medikamenten mit oder ohne ärztlichem Rezept, sowie bei allen stationären Behandlungen, gemäss Art. 42 Abs. 2 KVG, verwendet \cite{EidgenossischesDepartementdesInnern2017FaktenblattVergutungssysteme}.

Die Verarbeitung von Rechnungen, welche über das System Tier payant abgerechnet werden, kann der Versicherer, aufgrund der digitalen, strukturierten Daten, vollautomatisch gestalten.

Im Fall von Tiers garant stellt der Leistungserbringer die Rechnung direkt dem Patienten aus, welcher diese dann seinem Krankenversicherer zur Rückvergütung weiterleitet. Die Rechnung kann bei allen Krankenversicherern per Post und bei den meisten auch digital, im Kundenportal oder in der App, eingereicht werden \cite{EidgenossischesDepartementdesInnern2017FaktenblattVergutungssysteme} \todo{CITE für die Aussage Post und digitale einreichung}.

Rechnungen, welche per Post oder digital beim Versicherer zur Rückvergütung eingereicht werden, erreichen diesen in unterschiedlichster Qualität. Bei der Einreichung per Post kann die Qualität durch Kaffee-Flecken oder sonstige Beeinträchtigungen gemindert werden, der Versicherer kann aber viele andere Faktoren selbst beeinflussen. So kann er beispielsweise hochauflösende Scanner und optimale Beleuchtung einsetzen \todo{Zu beginn des Absatzes etwas viel "einreichen"} \todo{CITE, Experten-Interview? Persönliches Gespräch?}.

\todo[inline]{
Frage an Herrn Zenklusen:

Der obige Absatz wurde nach meinem Wissen durch meine Arbeit bei der AXA verfasst. Wie kann ich diesen mit einem Zitat untermauern? Ein Klassenkamerade hat mir Gesagt, ich könne ein "Persönliches Gespräch" mit einer Fachperson "zitieren", ohne dieses im Anhang transkribieren zu müssen, stimmt das?
}

Problematischer sind Rechnungen, welche digital an den Versicherer übermittelt werden. Wird ein Foto einer Rechnung über das Kundenportal eingereicht, so hat der Versicherer nur noch sehr wenig Einfluss auf die Qualität der Aufnahme. Schlechte Belichtung, kleine Auflösung und abgeschnittene Rechnungen sind nur wenige der Probleme, mit welchen der Versicherer zu kämpfen hat.

Egal wie und in welcher Qualität eine Rechnung einen Versicherer erreicht hat, muss dieser die Rechnung in eine elektronische, strukturierte Form bringen, damit diese dann durch ein Regelwerk verarbeitet werden kann. Dieser Vorgang wird Indexieren genannt. Viele Versicherer haben für die Indexierung bereits in Optical Character Recognition (OCR) Technologie investiert oder die Indexierung an eine externe Firma ausgelagert. 

Es gibt diverse Anbieter, welche diese Technologien oder die gesamte Indexierung als Service anbieten. Ein grosser Teil der Rechnungen muss aber, aus verschiedenen Gründen, nach wie vor manuell bearbeitet werden \todo{Quelle für manuelle arbeiten}. Dies beinhaltet sowohl die Nachbearbeitung nach der elektronischen Indexierung sowohl auch die komplett manuelle Indexierung. Die CSS Krankenkasse beschäftigt beispielsweise rund 200 Personen für diese manuelle Indexierung. Bei der Assura sind es rund 500 Personen \todo{Quelle für die Anzahl personen}.

\todo[inline]{
Hier noch etwas ausführen zu den hohen Kosten
}

% Im Jahr 2016 wiesen die Grundversicherer einen durchschnittlichen Verwaltungskostensatz von 4.7\% aus. Dies bedeutet  Ein guter Verwaltungskostensatz konnte die CSS Kranken-Versicherung AG im Jahr 2017 ausweisen weist im Jahr Die Indexierung dieser Rechnungen ist einer der Faktoren, welche auf die hohen Verwaltungskosten der Krankenversicherer schlägt. Es müssen 

Auch die AXA, eine internationale Versicherungsgesellschaft, sieht sich vor der Herausforderung der Indexierung von Rechnungen. Im Jahr 2017 lancierte die AXA eine Kranken-Zusatzversicherung. Neben der Zusatzversicherungen selbst bietet die AXA ihren Kunden einen Rechnungs-Weiterleitungs-Service. Das bedeutet, alle Rechnungen können der AXA gesendet werden. Rechnungen beziehungsweise Rechnungspoisitionen, welche die Zusatzversicherung betreffen, werden von der AXA vergütet und Rechnungspositionen, welche die Grundversicherung betreffen, werden zur Vergütung an den Grundversicherer weitergeleitet \cite{finanzen.ch2017AxaGewinnen}.

Ziel der AXA ist es, bis im Jahr 2020 insgesamt 100'000 Kunden zu gewinnen \cite{finanzen.ch2017AxaGewinnen}. Neben den Rechnungen, welche diese Kunden für Ihre AXA Zusatzversicherung einreichen, werden auch Rechnungen eingereicht, welche an den Grundversicherer weitergeleitet werden müssen. Die Anzahl an Rechnungen, welche die AXA verarbeiten muss, wird sich mit jedem gewonnen Kunden erhöhen.

\todo[inline]{
Aus vorhergehendem Paragraphen noch eine bessere Aussage zur Masse der Rechnungen machen. Gibt es Branchen-Durchschnitts-Werte an Anzahl Rechnungen pro Person?
}

Um die eingereichten Rechnungen zu verarbeiten, egal ob diese von der AXA selbst bezahlt oder an den Grundversicherer weitergeleitet werden, müssen diese Indexiert werden. Die Indexierung wird aktuell von einem externen Provider übernommen und ist eine Blackbox. Es ist allerdings bekannt, dass ein grosser Teil der Arbeiten, nach einem automatisierten OCR Schritt, manuell gemacht werden. Die aktuelle Indexierung birgt folgende zwei Probleme. 

\begin{itemize}
    \item Die Qualität der Indexierten Daten ist ungenügend. Fehler in der Indexierung, ein Medikament wurde beispielsweise als 1g anstelle 500mg Tablette indexiert, führen zu Fehlern in den Abrechnungen, welche im schlimmsten Fall eine Benachteiligung des Kunden und somit eine Reklamation verursachen. 
    \item Der hohe Anteil an manueller Arbeit verursacht einerseits hohe Kosten und ist andererseits nicht schnell genug oder gar nicht skalierbar.
\end{itemize}

Anhand der Problemstellung der Indexierung der eingehenden Rechnungen soll diskutiert werden, ob ein teilweiser oder gänzlich manueller Arbeitsschritt eines Kundenservices durch die Anwendung von künstlicher Intelligenz automatisiert werden kann.

Für die Problemstellung ist es nicht nur von relevant ob sondern auch in welcher Qualität dieser Arbeitsschritt automatisiert werden kann. Die Qualität stellt ein wichtiger Punkt dar, da schlechte Qualität ein Image-Schaden oder gar regulatorische Konsequenzen nach sich ziehen könnte \todo{CITE regulatorische Konsequenzen?}.

\subsection{Künstliche Intelligenz in anderen Branchen}

\todo[inline]{
Fallbeispiel als Start in die Einleitung und dann ab hier die Problemstellung verallgemeinern
}

\subsection{Forschungsstand}

\todo[inline]{
der Wissensstand im Bereich des Forschungsproblems (was weiss man bereits, was nochnicht) knapp beschrieben
}

\subsection{Forschungsfrage}

Aus dem beschriebenen Fallbeispiel und der Verallgemeinerung dessen wird folgende Forschungsfrage abgeleitet.

{
    \medskip
    \setlength{\fboxsep}{1em}
    \noindent\fcolorbox{igloo-darker}{igloo}{%
        \minipage[t]{\linewidth-2\fboxsep-2\fboxrule\relax}
            \begin{flushleft}
                \centering
                Können Kundenservices durch Künstliche Intelligenz automatisiert werden?
            \end{flushleft}
        \endminipage}
    \medskip
}

Weiter werden folgende Unterfragen abgeleitet.

\todo[inline]{
Unterfragen ausarbeiten...
}

\begin{itemize}
    \item Lohnt es sich für die AXA in die Automatisierung durch die Anwendung von künstlicher Intelligenz zu investieren? \todo{Müsste zur Beantwortung nicht eine ganze Kosten-Nutzen Rechnung gemacht werden?}
    \item Wäre eine Investition im Fallbeispiel durch etwaige Digitalisierungsvorhaben des Bundes gefährdet? \todo{relevant? sinnvoll?}
\end{itemize}


\vspace{2cm}
\todo[inline]{Missing a lot of citations in this chapter}
\todo[inline]{Soll noch auf Digitalisierungsvorhaben (wie TARMED, Patientendossier) eingegangen werden?}

\todo[inline]{
    Aktuelle Ansätze in der Branche
    
    \begin{itemize}
        \item App-Lösungen zur Einreichung der Rechnungen auf sehr unterschiedlichem Niveau
    \end{itemize}
    
    Problematik in anderen Branchen
    
    \begin{itemize}
        \item  Wo gibt es automatisierte Rechnungsverarbeitung?
        \item  ESR Scanning vom e-Banking
        \item  DropBox macht automatische indexierung von dokumenten
        \item  MS OfficeLens macht gute rectification
    \end{itemize}
}

\newpage
\section{Zielsetzungen, inhaltliche Abgrenzung}

\todo[inline]{
In der Zielsetzung werden neben der Beantwortung der Forschungsfrage die darüber hinausgehenden Ziele benennt, die mit der Thesis verfolgt werden (was soll mit der Untersuchung erreicht werden, wer kann welchen Nutzen aus der Thesis ziehen?). 

Indem angegeben wird, was (Forschungsfrage) warum (Zielsetzung) untersucht werden soll, kann auch definiert werden, welche Fragen, Inhalte und Ziele in der Thesis nicht verfolgt, also bewusst ausgeklammert werden.
}

Neben der Beantwortung der Forschungsfrage anhand des beschriebenen Fallbeispiels soll für die AXA geklärt werden, ob Investitionen in die Automatisierung des dargelegten Prozesses durch Künstliche Intelligenz gemacht werden sollen.

\todo[inline]{
Mehr...
}

\subsection{Abgrenzung Fallbeispiel}

Für das Fallbeispiel wird ein Prototyp entwickelt, welcher die Indexierung der Rechnungen abdeckt. Für den Prototypen ist es als Ausgangslage wichtig, in welcher Form und Qualität die Rechnungen bei der AXA ankommen. Weiter ist es für die Zielsetzung wichtig, in welches Format die Rechnungen gebracht werden müssen, damit diese anschliessend weiterverarbeitet werden können. Diese beiden Aspekte sollen als Rahmenbedingungen für den Prototypen gelten.

Die Weiterverarbeitung der Rechnungen nach der Indexierung, sprich die Auswertung, ob und wie eine bestimmte Rechnungsposition versichert ist, ist nicht Teil des Prototypen.

\todo[inline]{Definition, welche Attribute für die Verarbeitung interessant sind. Ist das für das Grobkonzept bereits wichtig? Eher nicht}

\newpage
\section{Methodische Vorgehensweise}

\todo[inline]{
Hier wird die methodische Vorgehensweise zur Beantwortung der Forschungsfrage erläutert. Die Vorgehensweise bezieht sich auf die Art der empirischen Datenerhebung (qualitativ, quantitativ oder eine Mischform) wie auch auf die geplante Auswertung der erhobenen Daten. Methoden der Datenerhebung sind beispielsweise eine Umfrage oder Interviews, Methoden der Datenanalyse sind beispielsweise statistische Tests oder eine Inhaltsanalyse. Es wird beschrieben, wie bei der Datenerhebung und -analyse vorgegangen werden soll (wer soll wie befragt werden, wie werden die Daten analysiert) und welche kritischen Aspekte in der Erhebung und Analyse zu erkennen sind
}

Während die Forschungsfrage aufgrund existierender Literatur diskutiert wird, bildet der Prototyp, welcher zur Diskussion des Fallbeispiels entwickelt wird, eine zentrale Rolle bei der Beantwortung der Forschungsfrage.

Damit der Prototyp und dessen Erfolg bewertet werden kann, werden zuerst die Rahmenbedingungen und Erfolgskriterien definiert. Es wird definiert welche Kriterien die einzulesenden Rechnungen erfüllen und in welchen Variationen diese daher kommen. Weiter wird definiert welche Daten in welcher Qualität für die Weiterverarbeitung der Rechnungen durch den Prototypen gewonnen werden müssen. Zur Messung der Qualität wird ein klares Vorgehen bestimmt.

Sind die Rahmenbedingungen geklärt, wird der Prototyp mit einem Set an Trainings-Rechnungen trainiert und mit einem Set an Test-Rechnungen getestet. Mit den Test-Rechnungen wird ermittelt, wieviele Fehler der Prototyp macht und ob dies im erwarteten Rahmen ist.

Die Resultate aus dem Prototypen werden diskutiert und Verbesserungspotential wird aufgezeigt. Es wird weiter erarbeitet, ob der Prototyp zu einer praktikablen Lösung weiterentwickelt werden soll oder nicht.

\todo[inline]{Überarbeiten}
 
\newpage
\section{Provisorisches Inhaltsverzeichnis} 

\todo[inline]{
Im provisorischen Inhaltsverzeichnis werden die thematischen Schwerpunkte der Thesis und welche wissenschaftlichen Theorien und Erklärungsansätze zur Beantwortung der Forschungsfrage herangezogen werden, definiert. Die Überschriften der Hauptkapitel der Thesis lassen die relevanten Teilaspekte des Themas und das methodische Vorgehen erkennen. Weitere Hinweise zum Aufbau der Thesis befinden sich in den Richtlinien für die Erstellung von Bachelor und Master Theses, Punkt 5.
}

{
    \renewcommand\labelitemi{--}
    \renewcommand{\labelenumi}{\arabic{enumi}}
    \renewcommand{\labelenumii}{\labelenumi.\arabic{enumii}}
    \renewcommand{\labelenumiii}{\labelenumii.\arabic{enumiii}}
    \begin{itemize}[topsep=0pt,itemsep=2pt,partopsep=4pt, parsep=4pt]
        \item Management Summary
        \item Ehrenwörtliche Erklärung
        \item Abkürzungsverzeichnis
        \item Einleitung
    \end{itemize}
    \begin{enumerate}[topsep=0pt,itemsep=2pt,partopsep=4pt, parsep=4pt]
        \item Grundlagen der künstlichen Intelligenz
        \begin{enumerate}[topsep=0pt,itemsep=2pt,partopsep=4pt, parsep=4pt]
            \item Neuronale Netzwerke
            \begin{enumerate}[topsep=0pt,itemsep=2pt,partopsep=4pt, parsep=4pt]
                \item Convolutional Neuronal Networks
                \item Recurrent Neuronal Networks
                \item Long-Short-Term-Memory Networks
            \end{enumerate}
            \item Natural Language Processing
            \begin{enumerate}[topsep=0pt,itemsep=2pt,partopsep=4pt, parsep=4pt]
                \item Named Entity Recognition
                \item Text classification
            \end{enumerate}
        \end{enumerate}
        \item Künstliche Intelligenz in der Automatisierung (Literatur)
        \item Entwicklung eines Prototypen zur Indexierung von Rechnungen (Empirisch/Praktisch)
        \begin{enumerate}[topsep=0pt,itemsep=2pt,partopsep=4pt, parsep=4pt]
            \item Vorgehen
            \item Rahmenbedingungen
            \begin{enumerate}[topsep=0pt,itemsep=2pt,partopsep=4pt, parsep=4pt]
                \item Format und Qualität der Rechnungen
                \item Erwartetes Format und Qualität der Resultate
            \end{enumerate}
            \item Texterkennung durch LSTM Netzwerk
            \item Optimierung der Resultate der Texterkennung
            \begin{enumerate}[topsep=0pt,itemsep=2pt,partopsep=4pt, parsep=4pt]
                \item Rechtschreib- und Grammatik-Korrektur mit einem LSTM Netzwerk
                \item ...
            \end{enumerate}
            \item Extraktion von strukturierten Informationen
            \begin{enumerate}[topsep=0pt,itemsep=2pt,partopsep=4pt, parsep=4pt]
                \item Named Entity Recognition
                \item Text classificiation
                \item Standardisierte Formate \todo{Besserer Titel, extraktion ohne AI durch bekannte Formate (E.g. TARMED)}
            \end{enumerate}
            \item Zusammenfassung der Resultate
        \end{enumerate}
        \item Diskussion
        \item Anhang
        \begin{enumerate}[topsep=0pt,itemsep=2pt,partopsep=4pt, parsep=4pt]
            \item Literaturverzeichnis
            \item Abbildungsverzeichnis
            \item Tabellenverzeichnis
        \end{enumerate}
    \end{enumerate}
}

\todo[inline]{Die ganze Struktur ist etwas tief Verschachtelt}
\todo[inline]{Titel sind etwas langweilig}

\newpage
\section{Meilensteine}

\todo[inline]{
Hier werden die wichtigsten Arbeitsschritte vom Erstellen des Grobkonzeptes bis zur Abgabe der Bachelor Thesis als Meilensteine wiedergegeben und der Betreuungsperson 3 Besprechungstermine vorgeschlagen (siehe Punkt 2 der Spezifischen Regelungen für Bachelor Theses). Die Betreuungsperson bestätigt bei der Prüfung des Grobkonzeptes die Terminvorschläge oder schlägt andere Termine vor.
}
 
\begin{center}
    \renewcommand{\arraystretch}{1.25}
    \setlength{\tabcolsep}{15pt}
    \begin{tabular}{ | p{8cm} | l |}
    \hline
    \rowcolor{ccc} Was? & Wann? \\ \hline
    1. Besprechungstermin auf Basis des ersten Entwurfs des Grobkonzeptes. & 13. November 2018 \\ \hline
    
    Literaturrecherche zum gewählten Thema fortführen und Grobkonzept finalisieren. & 28. November 2018 \\ \hline
    
    \rowcolor{orange} Eingabe des Namens der Betreuungsperson, des Grobkonzeptes, des Titels, und gegebenenfalls der Angaben zur externen Fachperson & 2. Dezember 2018 \\ \hline
    
    Prüfung der Betreuungsanfrage, des Titels und des Grobkonzeptes durch die Betreuungsperson & 17. Dezember 2018 \\ \hline
    
    Theorieteil verfassen und empirische Untersuchung vorbereiten & TODO \\ \hline
    
    2. Besprechungstermin nach Fertigstellung des Theorieteils und zur Gestaltung des Messinstrumentes zwecks Datenerhebung & TODO \\ \hline
    
    Empirische Untersuchung durchführen & TODO \\ \hline
    
    3. Besprechungstermin zur vorläufigen Beantwortung der Forschungsfrage & TODO \\ \hline
    
    Empirischer Teil finalisieren & 29. März 2019 \\ \hline
    
    Fertigstellung der gesamten Thesis & 1. April 2019 \\ \hline
    
    Korrektorat der Thesis durchführen und Feedback einarbeiten, Thesis drucken und binden lassen & 26. April 2019 \\ \hline
    
    \rowcolor{orange} Frist zur Abgabe und Hochladen der Bachelor Thesis & 3. Mai 2019 18:00 Uhr \\ \hline
    
    \end{tabular}
\end{center}


\newpage
\section{Erste Quellenverweise zum Thema}
\todo[inline]{
Hier werden bereits gelesene Quellen angeführt und auf weiterführende Literatur verwiesen, die noch ausgewertet wird.
}

\todo[inline, color=red]{TODO}

\newpage

\section{Notizen - Outline}

\subsection{Einleitung}
\begin{itemize}
    \item Gesundheitssystem ist grosser Kostenverursacher
    \item TG vs TP
    \begin{itemize}
         \item KVG vs. VVG
    \end{itemize}
	 \item Krankenkassen tragen mehr als 1/3 der Kosten
	 \item Grosser Verwaltungsaufwand bei den Krankenkassen
     \begin{itemize}
        \item Dadurch auch aktuelle Diskussion über Einführung Einheitskasse (relevant?)
        \item Aufzeigen der entstehenden Kosten?!
    \end{itemize}
	\item Grundversicherung stark reglementiert
    \begin{itemize}
        \item wenig Innovationsdruck/-potential
        \item Wenig Abhebung von der Konkurrenz
        \begin{itemize}
			\item Keiner macht Zusatzleistung, höchstens Abstriche (e.g. Apotheken der Assura)
        \end{itemize}
        \item Schleichender Wettbewerb
        \item Preisdifferenzierung ist enorm wichtig -> Begründung warum Forschungsfrage interessant
    \end{itemize}
	\item Digitalisierungsvorhaben des Bundes nicht erfolgreich -> Begründung warum Forschungsfrage interessant
    \begin{itemize}
        \item TARMED, SwissMedic, PatientenDossier, …
    \end{itemize}
    \item Kunde erwartet sein Geld möglichst schnell zurück
    \begin{itemize}
        \item Automatisierte verarbeitung der Rechnung erlaubt dies  -> Begründung warum Forschungsfrage interessant
    \end{itemize}
    \item Forschungsfrage
    \begin{itemize}
        \item Abgrenzung
    \end{itemize}
\end{itemize}
	
	
\subsection{Theoretischer Teil}
Was ist Machine Learning/Deep Learning?

Supervised vs Un-supervised learning?

Welche Anwendungsbereiche gibt es aktuell?

\begin{itemize}
    \item LSTM
    \begin{itemize}
        \item Einführung in Neuronale Netzwerke
        \item Was ist ein RNN/LSTM und wie funktioniert diese Art von NN
        \item Vergleich mit CNN
        \item Faktoren für den Erfolg eines NN
        \item Training
        \begin{itemize}
            \item Merkmale geeigneter Trainingsdaten
            \item Problemstellungen
            \begin{itemize}
                \item Over-fitting
            \end{itemize}
        \end{itemize}
        \end{itemize}
		\item Anwendungsgebiete für LSTM Modelle
        \begin{itemize}
	        \item OCR
            \begin{itemize}
			    \item Pre-condition für erfolgreiches OCR
                \begin{itemize}
				    \item Gute Bildqualität
                    %\begin{itemize}
					    \item Aktuell nicht gegeben
						\item Input Kanal optimieren
                        %\begin{itemize}
							\item Verbesserung des Upload Prozess des im Kundenportal
                            %\begin{itemize}
								\item Wie viele Rechnungen werden Digital eingereicht? (aktuell ca. 2/3 digital)
                            %\end{itemize}
                        %\end{itemize}
						\item Image rectification
                        %\begin{itemize}
							 \item leptonica
							 \item OpenCV
                        %\end{itemize}
                    %\end{itemize}
					\item Eine Rechnung pro Upload
                    %\begin{itemize}
						\item Aktuell nicht gegeben
						\item TARMED einfach erkennbar
                    %\end{itemize}
                \end{itemize}				
			\item Machine translation
			\item Grammatikprüfung
			\item Image recognition
        \begin{itemize}
            \item Nicht wirklich relevant für diese Arbeit
        \end{itemize}
		\item Text Classification
        \begin{itemize}
			\item Vergleich mit klassischem NLP?
			\item Warum LSTM und nicht klassisches NLP?
        \end{itemize}
        \end{itemize}
		    \item Named Entity Recognition (NER)
        \begin{itemize}
			\item Erkennung von Namen
			\item NLP vs Machine Learning basiert
        \end{itemize}
    \end{itemize}
	\item Struktur-Anforderungen der Daten für die Weiterverarbeitung
    \begin{itemize}
		\item Syrius requirements
		\item Fachliche Anforderungen
        \begin{itemize}
    		\item KVG vs. VVG beachten
        \end{itemize}
    \end{itemize}
    \item Lösungen in anderen Branchen
    \item Lösungen von anderen Unternehmen
    \begin{itemize}
		\item Man-power
        \begin{itemize}
		    \item Assura 500 / CSS 300
            \begin{itemize}
			    \item TODO: Diese Daten sind nur "Hörensagen"
            \end{itemize}
        \end{itemize}
    \end{itemize}
\end{itemize}


\subsection{Empirischer Teil}
\begin{itemize}
    \item Methodik
    \begin{itemize}
		\item Prototyping
		\item Messkriterien des Prototypen
		\item Begrüdnung des Vorgehens
    \end{itemize}
	\item Prototyp
    \begin{itemize}
		\item OCR mittels Tesseract
        \begin{itemize}
			\item Trainieren der LSTM engine mittels artificial training data aus TARMED datenbank
			\item Artificial data wird analog eingereichter Rechnungen gemacht
            \begin{itemize}
				\item Schrfitart
				\item Noise
				\item Verzerrung
            \end{itemize}
        \end{itemize}
	    \item Spell Checking mittels eigenem LSTM Modell
        \begin{itemize}
		    \item Inspiration an DeepSpell, etc.
		    \item Training des Modells anhand artificial spelling errors basierend auf
            \begin{itemize}
				\item Google Training Datasets
				\item TARMED tarifen
				\item SYR Codes
				\item EMR Methods
				\item Other Methods
				\item Custom Tarifziffern
				\item Care Providers (Names, Addresses, …)
				\item Real Invoices
				\item Medizinischen Texten / Anamnese
            \end{itemize}
        \end{itemize}
    \end{itemize}
    \item Noise reduction
    \begin{itemize}
	    \item Pre and/or Post OCR
        \item Post OCR
        \begin{itemize}
	    	\item NN with input word + confidence -> output: noise | no-noise
        \end{itemize}
    \end{itemize}
    \item Erkennung von mehreren Rechnungen in einem Upload
    \begin{itemize}
	    \item Mehrere auf einem Bild
	    \item Mehrere auf mehreren Bildern
    \end{itemize}
    \item Klassifizierung von Rechnungen
    \begin{itemize}
		\item Traditioneller Ansatz für TARMED ja/nein
		\item LSTM Text Classification für "Prosa" (non-TARMED) Rechnungen
    \end{itemize}
    \item Strukturieren von Daten
    \begin{itemize}
		\item TARMED gut strukturierbar
		\item Für andere Rechnungen wird’s etwas schwieriger
        \begin{itemize}
			\item Patienten Erkennung
			\item Care Provider Erkennung
			\item Tabellen Erkennung
            \begin{itemize}
			    \item Erkennung der einzelnen Positionen
            \end{itemize}
        \end{itemize}
        \item Business Rules zur automatisierten Verarbeitung    
    \end{itemize}  
\end{itemize}

\subsection{Schlussfolgerungen und Empfehlungen}
\begin{itemize}
	\item Bewertung des Prototypen
	\item Beantwortung der Forschungsfrage
	\item Reflexion
	\item Empfehlungen / Zukunft
  \begin{itemize}
		\item Erweitern der Pipeline für andere Rechnungstypen (e.g. Fitnessabos)
		\item Erweitern der Business Rules
		\item On-Device OCR/NER um den upload Prozess zu verbessern
    \begin{itemize}
			\item Benutzer soll den Patientennamen nicht eingeben müssen, er soll erkannt werden
			\item Benutzer kann die erhobenen Daten direkt gegenprüfen
Bspw. Behandlungsdatum, etc.
    \end{itemize}
  \end{itemize}
\end{itemize}
 
\section{Notizen - sonstiges}
 
\subsection{Lösungsansatz ``Input Pipeline''}
\begin{itemize}
    \item Erste Optimierung bereits beim Kunden, da dort ``Crowd Sourcing'' betrieben werden kann
    \begin{itemize}
        \item Image rectification mit anpassungsmöglichkeit durch Enduser (analog OfficeLens)
        \item Quality Checking
    \end{itemize}
    \item Optimierung auf der Service Seite
    \begin{itemize}
        \item Gray Scale
    \end{itemize}
    \item OCR
    \begin{itemize}
        \item LSTM basierter Ansatz
        \begin{itemize}
            \item Vergleich zur feature detection basierten vorgehensweise (Stats, warum was besser ist)
        \end{itemize}
    \end{itemize}
\end{itemize}

Machine Learning ist sehr breit
  - Text Mining
  - Deep Learning
  - OCR
  - Classification
  - Image optimization
  
\newpage
\section{Anhang}

\subsection{Literaturverzeichnis}
\todo[inline]{Add "Abgerufen am ... von ..." to web page citations}
{
    % This renewcommand hides the title of the bibliography
    \renewcommand{\section}[2]{}%
    \bibliography{references}
}

\todo[inline]{Why the heck is there a "Literatur" heading in the toc}

\newpage
\subsection{Abbildungsverzeichnis}
{
    % This renewcommand hides the title of the listoffigures
    \renewcommand{\section}[2]{}%
    \listoffigures
}

\newpage
\subsection{Tabellenverzeichnis}
{
    % This renewcommand hides the title of the listoftables
    \renewcommand{\section}[2]{}%
    \listoftables
}

\end{document}