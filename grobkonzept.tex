% font-size as in "Richtlinien Semesterarbeit"
\documentclass[12pt, twoside, table]{extarticle}
% TODOs
\usepackage[color=lightgray]{todonotes}
% colored tables
\usepackage{colortbl}
\usepackage{xcolor}
% Set margin analogous to word template
\usepackage[left=2.5cm, right=2.5cm, top=2.5cm, bottom=2cm]{geometry}
% line-height as in "Richtilinien Semesterarbeit"
\renewcommand{\baselinestretch}{1.15}
% custom paragraph spacing
\setlength{\parskip}{0.25em}
% citation style
\usepackage{apacite}
\bibliographystyle{apacite}
% plumbing...
\usepackage[utf8]{inputenc}
\usepackage[german]{babel}
% unnumbered sections 
\newcommand{\sectionUnnumbered}[1] {
  \section*{#1}
  \addcontentsline{toc}{section}{#1}
}
% custom colors
\definecolor{ccc}{rgb}{0.8,0.8,0.8}

\title{Grobkonzept}
\author{Sven Tschui}
\date{October 2018}

\begin{document}

\begin{titlepage}
    {
    	\centering
    	
    	\vspace*{2cm}
    	% \includegraphics[width=0.15\textwidth]{example-image-1x1}\par\vspace{1cm}
    	{\Large\fontfamily{ppl}\selectfont Grobkonzept zur Bachelor Thesis}
    	
    	{\LARGE\bfseries\fontfamily{ppl}\selectfont Automatisierte Rechnungsverarbeitung in der Krankenversicherung mittels Deep Learning \par}
    	
    	\vspace{3cm}
    	
    	{Zürcher Fachhochschule\par}
    	
    	{\bfseries\large\fontfamily{ppl}\selectfont HWZ Hochschule für Wirtschaft Zürich\par}
    	
    	\vfill
    }
    {
        \renewcommand{\arraystretch}{1.5}
        \setlength{\tabcolsep}{0pt}
        \begin{flushleft}
    	\begin{tabular}{ l@{\hspace{1.5cm}} l }
         Student: & Sven Tschui \\
         Studiengruppe: & BWI-A15 \\
         Betreuungsperson: & Dr. Oliver Zenklusen \\
         Datum & 1. Dezember 2018 \\
        \end{tabular}
        \end{flushleft}
    }
\end{titlepage}

\newpage
\tableofcontents

\section{Ausgangslage, Forschungsproblem und -frage}

\todo[inline]{
In der Ausgangslage wird das Thema zuerst allgemein vorgestellt, dann wird auf einen bestimmten Teilaspekt des Themengebiets fokussiert. 

Diese Fokussierung führt zum Forschungsproblem und damit zu den Erkenntnissen, die gewonnen werden sollen. Gründe werden aufgeführt, weshalb es relevant ist, das gewählte Problem zu untersuchen. Ausserdem wird der Wissensstand im Bereich des Forschungsproblems (was weiss man bereits, was noch nicht) knapp beschrieben. 

Die Forschungsfrage schliesslich bündelt die zentralen Aspekte des Forschungsproblems als zugespitzte Frage. Die Frage sollte bereits so konkret sein, dass sie in einer Thesis untersucht werden kann.

\textit{(Bitte diesen Text jeweils nicht löschen. Er dient als Information für die Betreuungsperson.)}
}

\todo[inline, color=red]{
    Alle Aspekte der Anforderungen abdecken. Bspw. 
    
    - das Thema zuerst allgemein vorgestellt
    
    -  Gründe werden aufgeführt, weshalb es relevant ist, das gewählte Problem zu untersuchen.
    
    - der Wissensstand im Bereich des Forschungsproblems (was weiss man bereits, was noch nicht) knapp beschrieben
    
    - Forschungsfrage schliesslich bündelt die zentralen Aspekte des Forschungsproblems
}

Schneller und guter Kundenservice ist wichtig um Kunden halten zu können, dies ist längst bekannt \todo{CITE}. Auch die Auswirkungen einer hohen Kundenzufriedenheit auf die Kundenloyalität werden als sehr stark anerkannt \todo{CITE}. Aus diesen Gründen investieren Unternehmen immer stärker in diese Services und versuchen sich damit von der Konkurrenz zu differenzieren \todo{CITE}. Diese Services bereitzustellen kann für ein Unternehmen sehr kostenintensiv sein \todo{CITE}. \todo[inline, color=orange]{ Fakten und Zahlen würden diese Aussage nicht nur untermauern sondern auch spannender machen} Kundenservices sollen heutzutage mit geeigneter Soft- und Hardware automatisiert werden, um die Kosten so gering wie möglich zu halten \todo{CITE}. Die Automatisierung hat ausserdem den Effekt, dass Services schneller abgewickelt werden können, was im Zeitalter der ungeduldigen Kunden ein wichtiger Differenzierungsfaktor sein kann \todo{CITE}. Die Automatisierung durch klassische Software stösst aber teilweise \todo{konkrekt sagen, wann diese Grenzen auftreten} an ihre Grenzen. Die klassische, strukturierte Programmierweise bildet Logik ab, welche durch Ursache-Wirkung klar definierbar ist: Wenn eine Bedingung eintrifft, wird etwas ausgeführt \todo{QUOTE}. Ein Abwägen von Fall zu Fall, wie dies ein Angestellter im Kundenservice machen könnte, ist mit solcher Software nicht möglich. Eine Lösung für diese Limitierung bietet die künstliche Intelligenz: Der Computer wird nicht mehr angewiesen was zu tun ist, sondern was das Ziel ist. Der Computer wird dann auf die erreichung dieses Ziels trainiert \todo{QUOTE}.

In dieser Arbeit werden die Möglichkeiten diskutiert, welche die künstliche Intelligenz für die Automatisierung von Kundenservices bietet. Im ersten Teil werden die grundlegenden Elemente der Künstlichen Intelligenz erläutert und Herausforderungen in der Automatisierung der Kundenservices aufgezeigt. \todo[inline]{weiterer Inhalt, e.g. theoretische erfahrung, Praxisbeispiele} Der zweite Teil widmet sich einem Fallbeispiel eines Kundenservices, welcher in einem Prototypen durch den Einsatz von Künstlicher Intelligenz automatisiert werden soll. Der Erfolg des Experiments wird anhand zuvor definierten Erfolgskriterien diskutiert.

\subsection{Fallbeispiel}

In der Schweiz beliefen sich die Kosten für das Gesundheitswesen im Jahr 2015 auf 77.8 Milliarden Franken. Über 35\% dieser Kosten wurden durch die obligatorische Krankenversicherung gedeckt. Weitere knapp 7\% wurden durch Zusatzversicherungen übernommen. Die Krankenversicherer finanzierten somit mit knapp 42\% einen beträchtlichen Teil des Gesundheitswesens in der Schweiz \cite{BundesamtfurStatistik2018Finanzierung, BundesamtfurStatistik2017KostenDaten}.

Die Kosten des Gesundheitswesen steigen stetig an, so weisen die Zahlen vom Jahr 2016 bereits Kosten von über 80 Milliarden Franken nach. Auch in den folgenden Jahren sollen die Kosten weiter steigen \cite{BundesamtfurStatistik2018Finanzierung}.

Die Kosten, welche die Krankenversicherer tragen, werden mit einem von zwei Systemen, Tiers payant oder Tiers garant, vergütet \cite{EidgenossischesDepartementdesInnern2017FaktenblattVergutungssysteme}. 

Beim System Tier payant belastet der Leistungserbringenr (bspw. Arzt oder Apotheke) die Kosten direkt dem Krankenversicherer. Dies geschieht, in dem der Patient mit seiner Kranken\-versicherungs-Karte bezahlt. Anhand dieser Karte, welche vom Krankenversicherer ausgestellt wird, können Deckungen für den Patienten überprüft sowie die Rechnung direkt an den Versicherer übermittelt werden. In diesem Fall wird die Rechnung bereits in digitaler, strukturierter Form übermittelt und der Krankenversicherer kann mit dem entsprechenden Regelwerk die Rechnung automatisch verarbeiten. Werden Kosten nicht vom Krankenversicherer getragen, weil beispielsweise ein Selbstbehalt vereinbart wurde, die Franchise noch nicht aufgebraucht ist oder der Patient für diese Behandlung gar nicht versichert ist, verrechnet der Versicherer die Kosten dem Patienten weiter \cite{EidgenossischesDepartementdesInnern2017FaktenblattVergutungssysteme}.

Häufig wird Tier payant in Apotheken verwendet, wenn beispielsweise Medikamente, mit oder ohne ärztlichem Rezept, gekauft werden. Auch Ärzte verwenden teilweise diese Methode \todo{Quelle}.

Im Fall von Tiers garant stellt der Leistungserbringer die Rechnung direkt dem Patienten aus, welcher diese dann seinem Krankenversicherer zur Rückvergütung zustellt. Die Rechnung kann bei allen Krankenversicherern per Post und bei den meisten auch digital, im Kundenportal oder in der App, eingereicht werden \cite{EidgenossischesDepartementdesInnern2017FaktenblattVergutungssysteme}.

Rechnungen, welche per Post oder digital beim Versicherer eingereicht werden, erreichen diesen in unterschiedlichster Qualität. Bei der Einreichung per Post kann die Qualität durch Kaffee-Flecken oder sonstige Beeinträchtigungen gemindert werden, der Versicherer kann aber viele andere Faktoren selbst beeinflussen. So kann er beispielsweise hochauflösende Scanner und optimale Beleuchtung einsetzen.

Problematischer sind Rechnungen, welche digital an den Versicherer übermittelt werden. Wird ein Foto einer Rechnung über das Kundenportal eingereicht, so hat der Versicherer nur noch sehr wenig Einfluss auf die Qualität der Aufnahme. Schlechte Belichtung, kleine Auflösung und abgeschnittene Rechnungen sind nur wenige der Probleme, mit welchen der Versicherer zu kämpfen hat.

Egal wie und in welcher Qualität eine Rechnung einen Versicherer erreicht hat, muss dieser die Rechnung in eine elektronische, strukturierte Form bringen, damit diese dann durch ein Regelwerk verarbeitet werden kann. Diesen Vorgang wird Indexieren genannt. Viele Versicherer haben für die Indexierung bereits in Optical Character Recognition (OCR) Technologie investiert. Es gibt diverse Anbieter, welche diese Technologien oder die gesamte Indexierung als Service anbieten. Ein grosser Teil der Rechnungen muss aber aus verschiedenen Gründen nach wie vor Manuell bearbeitet werden \todo{Quelle für manuelle arbeiten}. Dies beinhaltet sowohl die Nachbearbeitung nach der elektronischen Indexierung mittels OCR Technologie sowohl auch die komplett manuelle Indexierung. Die CSS Krankenkasse beschäftigt beispielsweise rund 200 Personen für diese manuelle Indexierung. Bei der Assura sind es gar run 500 Personen \todo{Quelle für die Anzahl personen}.

Die AXA ist eine Internationale Versicherungsgesellschaft mit Hauptsitz in Paris, welche auch in der Schweiz aktiv ist. Die Gesundheitsvorsorge ist für die gesamte AXA Gruppe ein strategisches Thema und so wurde im Jahr 2017 eine Krankenzusatzversicherung in der Schweiz lanciert \cite{finanzen.ch2017AxaGewinnen}. Die AXA bietet ihren aktuell 25'000 Kunden (Stand September 2018) \todo{Quelle für die Zahlen, sind die überhaupt öffentlich?} diverse Zusatzversicherungen im Gesundheitsbereich. Weiter übernimmt die AXA den Wechsel zum günstigsten Grundversicherer und die Weiterleitung der bei der AXA eingereichten Rechnungen an diesen Grundversicherer \cite{finanzen.ch2017AxaGewinnen}.

Um Rechnungen zu verarbeiten, egal ob diese von der AXA selbst bezahlt oder an den Grundversicherer weitergeleitet werden, müssen diese Indexiert werden. Die Indexierung wird aktuell von einem externen Provider erledigt und ist eine Blackbox. Es ist allerdings bekannt, dass ein grosser Teil der Arbeiten, nach einem automatisierten OCR Schritt, manuell gemacht werden. Die aktuelle Indexierung birgt verschiedene Probleme und Risiken. 

\begin{itemize}
    \item Die Qualität der Indexierten Daten ist ungenügend. Fehler in der Indexierung, ein Medikament wurde beispielsweise als 1g anstelle 500mg Tablette indexiert, führen zu Fehlern in den Abrechnungen, welche im schlimmsten Fall eine Benachteiligung des Kunden und somit eine Reklamation verursachen. 
    \item Der hohe Anteil an manueller Arbeit verursacht einerseits hohe Kosten und ist andererseits nicht schnell genug oder gar nicht skalierbar.
\end{itemize}

Die Indexierung der eingehenden Rechnung soll nun nicht nur automatisiert sondern Qualitativ verbessert werden. Eine Automatisierung dieses Prozessen mit hoher Qualität würde ein extremer Wettbewerbsvorteil für die AXA bedeuten, da, gegenüber der Konkurrenz, Kosten gespart werden könnten.  

In dieser Arbeit wird ein Prototyp entwickelt, welcher die Indexierung der Rechnungen mittels Künstlicher Intelligenz übernimmt. Anhand des Prototypen wird diskutiert, ob die Indexierung in dieser Art möglich und die resultierende Qualität ausreichend ist.

\subsection{Forschungsfrage}

Aus dem beschriebenen Forschungsproblem und Fallbeispiel wird folgende Forschungsfrage definiert.

{
    \medskip
    \setlength{\fboxsep}{1em}
    \noindent\fcolorbox{red}{yellow}{%
        \minipage[t]{\linewidth-2\fboxsep-2\fboxrule\relax}
            Können komplexe Kundenservices durch Künstliche Intelligenz automatisiert und dadurch ein Wettbewerbsvorteil erzielt werden?
        \endminipage}
    \medskip
}

Weiter werden folgende Unterfragen abgeleitet.

\begin{itemize}
    \item \todo[inline, color=red]{TODO}
    \item \todo[inline, color=red]{TODO}
    \item \todo[inline, color=red]{TODO}
\end{itemize}

\newpage

\todo[inline, color=red]{Missing a lot of citations in this chapter}


\todo[inline, color=orange]{
    
    \bfseries{NOTES}
    
    Ausgangslage Rechnungsverarbeitung im Krankenkassen Umfeld
    
    - Input Kanäle sind divers
    
      -- Post [ca. 40\%] (Scanning)
      
      -- Direkt [sehr wenig] (Arzt -> KK)
      
      -- E-Mail [wenig, soll vermieden werden] (PDF / Foto)
      
      -- Kundenportal [ca. 60\%] (PDF / Foto)
    
    - Qualität ist je nach Input Kanal sehr schlecht
    
      -- Fotos sind unscharf, verzerrt, abgeschnitten, unvollständig, papier in schlechtem zustand (gefaltet, "verchrugeled", kaffee flecken)
    
    - Rechnungen müssen einzeln Verarbeitet werden (Da jede Rechnung einzeln geprüft werden muss)
    
      -- Oft werden mehrere Rechnungen gesendet/hochgeladen
    
    - TG vs TP
    
    - Digitalisierungsvorhaben des Bundes nicht sehr erfolgreich
    
      -- swissmedic plante ursprüngliche digitalisierung, daraus entstand TARMED dokumente
    
      -- Patientendossier steht seit Jahren in den Kinderschuhen
    
      -- medidata.ch -> erfolgreich? Getrieben vom Bund oder privat?
    
    - TARMED kann durch standardisierung durch herkämmliche Zonal OCR verfahren verarbeitet werden
    
    - Nicht TARMED belege sind unstrukturiert und können aktuell noch nicht automatisiert verarbeitet werden
    
    - Im Bereich VVG ist TARMED nur ca. 50\%
    
    - AXA bietet nur VVG
    
    - Grosse Krankenkassen stellen sehr viele Personen an, um die Dokumente zu indexieren
    
    Aktuelle Ansätze in der Branche
    - App-Lösungen zur Einreichung der Rechnungen auf sehr unterschiedlichem Niveau
      - Saniats cooles app
      - Sympany nur Web Portal ohne Bild optimierung
    
    Problematik in anderen Branchen
    - Wo gibt es automatisierte Rechnungsverarbeitung?
    - ESR Scanning vom e-Banking
    - DropBox macht automatische indexierung von dokumenten
    - MS OfficeLens macht gute rectification
    
    Machine Learning bietet immer mehr Möglichkeit, IT Systeme intelligent zu machen und eröffnet neue chancen, bla bla bla
    
    \textbf{Forschungsfrage}: Kann die Leistungsverarbeitung im für Krankenversicherer mit Hilfe von Artificial Intelligence automatisiert werden?
}

\newpage
\section{Zielsetzungen, inhaltliche Abgrenzung}

\todo[inline]{
In der Zielsetzung werden neben der Beantwortung der Forschungsfrage die darüber hinausgehenden Ziele benennt, die mit der Thesis verfolgt werden (was soll mit der Untersuchung erreicht werden, wer kann welchen Nutzen aus der Thesis ziehen?). 

Indem angegeben wird, was (Forschungsfrage) warum (Zielsetzung) untersucht werden soll, kann auch definiert werden, welche Fragen, Inhalte und Ziele in der Thesis nicht verfolgt, also bewusst ausgeklammert werden.
}

In dieser Thesis wird die Forschungsfrage, ob Künstliche Intelligenz zur Automatisierung von Kundenservices angewendet werden kann, beantwortet. Dies wird anhand einem konkretem Fallbeispiel geklärt. Die Diskussion dieses Fallbeispiels soll für die AXA ausserdem klären, ob weitere Investitionen, in die Automatisierung des dargelegten Prozesses durch Künstliche Intelligenz, gemacht werden sollen.

Die Forschungsfrage ist für die AXA von höchstem Interesse, da ein starkes Wachstum angestrebt wird und somit geplant werden muss, ob in Zukunft die Anzahl der Mitarbeiter für diese Aufgaben erhöht werden muss oder nicht. Auch würde eine solche Automatisierung ein erheblicher Wettbewerbsvorteil darstellen.

\subsection{Abgrenzung Fallbeispiel}

Der Prototyp für das Fallbeispiel behandelt die Indexierung von Rechnungen. Es ist für den Prototypen relevant in welcher Form und Qualität die Rechnungen eingereicht werden, jedoch ist die optimierung dieser nicht Teil dieser Arbeit. Weiter ist es für den Prototypen relevant, in welcher Form und Qualität die Daten zur weiteren Verarbeitung aufbereitet werden müssen, jedoch ist die Weiterverarbeitung selbst nicht Teil dieser Arbeit.

\todo[inline, color=orange]{
    Definition, welche Attribute für die Verarbeitung interessant sind
}

\newpage
\section{Methodische Vorgehensweise}

\todo[inline]{
Hier wird die methodische Vorgehensweise zur Beantwortung der Forschungsfrage erläutert. Die Vorgehensweise bezieht sich auf die Art der empirischen Datenerhebung (qualitativ, quantitativ oder eine Mischform) wie auch auf die geplante Auswertung der erhobenen Daten. Methoden der Datenerhebung sind beispielsweise eine Umfrage oder Interviews, Methoden der Datenanalyse sind beispielsweise statistische Tests oder eine Inhaltsanalyse. Es wird beschrieben, wie bei der Datenerhebung und -analyse vorgegangen werden soll (wer soll wie befragt werden, wie werden die Daten analysiert) und welche kritischen Aspekte in der Erhebung und Analyse zu erkennen sind
}

Während die Forschungsfrage aufgrund existierender Literatur diskutiert wird, bildet der Prototyp, welcher zur Diskussion des Fallbeispiels entwickelt wird, eine zentrale Rolle bei der Beantwortung der Forschungsfrage.

Damit der Prototyp und dessen Erfolg bewertet werden kann, werden zuerst die Rahmenbedingungen und Erfolgskriterien definiert. Es wird definiert welche Kriterien die einzulesenden Rechnungen erfüllen und in welchen Variationen diese daher kommen. Weiter wird definiert welche Daten in welcher Qualität für die Weiterverarbeitung der Rechnungen durch den Prototypen gewonnen werden müssen. Zur Messung der Qualität wird ein klares Vorgehen bestimmt.

Sind die Rahmenbedingungen geklärt, wird der Prototyp mit einem Set an Trainings-Rechnungen trainiert und mit einem Set an Test-Rechnungen getestet. Mit den Test-Rechnungen wird ermittelt, wieviele Fehler der Prototyp macht und ob dies im erwarteten Rahmen ist.

Die Resultate aus dem Prototypen werden diskutiert und Verbesserungspotential wird aufgezeigt. Es wird weiter erarbeitet, ob der Prototyp zu einer praktikablen Lösung weiterentwickelt werden soll oder nicht.
 
\newpage
\section{Provisorisches Inhaltsverzeichnis} 

\todo[inline]{
Im provisorischen Inhaltsverzeichnis werden die thematischen Schwerpunkte der Thesis und welche wissenschaftlichen Theorien und Erklärungsansätze zur Beantwortung der Forschungsfrage herangezogen werden, definiert. Die Überschriften der Hauptkapitel der Thesis lassen die relevanten Teilaspekte des Themas und das methodische Vorgehen erkennen. Weitere Hinweise zum Aufbau der Thesis befinden sich in den Richtlinien für die Erstellung von Bachelor und Master Theses, Punkt 5.
}

{
    \renewcommand{\labelenumii}{\labelenumi\arabic{enumii}}
    \begin{itemize}
        \item Management Summary
        \item Ehrenwörtliche Erklärung
        \item Abkürzungsverzeichnis
        \item Einleitung
    \end{itemize}
    \begin{enumerate}
        \item Künstliche Intelligenz
        \begin{enumerate}
            \item Neuronale Netzwerke
            \item ...
        \end{enumerate}
        \item Automatisierung von Kundenservices
        \item Prototyp Rechnungseinreichung
        \begin{enumerate}
            \item Rahmenbedingungen
            \item Texterkennung
            \item Ergebnisverbesserung der Texterkennung
            \item Klassifizierung von Rechnungstypen
            \item Strukturierung der Rohdaten
            \item Erfolgscheck
        \end{enumerate}
        \item Anhang
        \begin{enumerate}
            \item Literaturverzeichnis
            \item Abbildungsverzeichnis
            \item Tabellenverzeichnis
        \end{enumerate}
    \end{enumerate}
}

\newpage
\section{Meilensteine}

\todo[inline]{
Hier werden die wichtigsten Arbeitsschritte vom Erstellen des Grobkonzeptes bis zur Abgabe der Bachelor Thesis als Meilensteine wiedergegeben und der Betreuungsperson 3 Besprechungstermine vorgeschlagen (siehe Punkt 2 der Spezifischen Regelungen für Bachelor Theses). Die Betreuungsperson bestätigt bei der Prüfung des Grobkonzeptes die Terminvorschläge oder schlägt andere Termine vor.
}
 
\begin{center}
    \renewcommand{\arraystretch}{1.25}
    \setlength{\tabcolsep}{15pt}
    \begin{tabular}{ | p{8cm} | l |}
    \hline
    \rowcolor{ccc} Was? & Wann? \\ \hline
    1. Besprechungstermin auf Basis des ersten Entwurfs des Grobkonzeptes. & 13. November 2018 \\ \hline
    
    Literaturrecherche zum gewählten Thema fortführen und Grobkonzept finalisieren. & 28. November 2018 \\ \hline
    
    \rowcolor{orange} Eingabe des Namens der Betreuungsperson, des Grobkonzeptes, des Titels, und gegebenenfalls der Angaben zur externen Fachperson & 2. Dezember 2018 \\ \hline
    
    Prüfung der Betreuungsanfrage, des Titels und des Grobkonzeptes durch die Betreuungsperson & 17. Dezember 2018 \\ \hline
    
    Theorieteil verfassen und empirische Untersuchung vorbereiten & TODO \\ \hline
    
    2. Besprechungstermin nach Fertigstellung des Theorieteils und zur Gestaltung des Messinstrumentes zwecks Datenerhebung & TODO \\ \hline
    
    Empirische Untersuchung durchführen & TODO \\ \hline
    
    3. Besprechungstermin zur vorläufigen Beantwortung der Forschungsfrage & TODO \\ \hline
    
    Empirischer Teil finalisieren & 29. März 2019 \\ \hline
    
    Fertigstellung der gesamten Thesis & 1. April 2019 \\ \hline
    
    Korrektorat der Thesis durchführen und Feedback einarbeiten, Thesis drucken und binden lassen & 26. April 2019 \\ \hline
    
    \rowcolor{orange} Frist zur Abgabe und Hochladen der Bachelor Thesis & 3. Mai 2019 18:00 Uhr \\ \hline
    
    \end{tabular}
\end{center}


\newpage
\section{Erste Quellenverweise zum Thema}
\todo[inline]{
Hier werden bereits gelesene Quellen angeführt und auf weiterführende Literatur verwiesen, die noch ausgewertet wird.
}

\todo[inline, color=red]{TODO}

\newpage

\section{Notizen - Outline}

\subsection{Einleitung}
\begin{itemize}
    \item Gesundheitssystem ist grosser Kostenverursacher
    \item TG vs TP
    \begin{itemize}
         \item KVG vs. VVG
    \end{itemize}
	 \item Krankenkassen tragen mehr als 1/3 der Kosten
	 \item Grosser Verwaltungsaufwand bei den Krankenkassen
     \begin{itemize}
        \item Dadurch auch aktuelle Diskussion über Einführung Einheitskasse (relevant?)
        \item Aufzeigen der entstehenden Kosten?!
    \end{itemize}
	\item Grundversicherung stark reglementiert
    \begin{itemize}
        \item wenig Innovationsdruck/-potential
        \item Wenig Abhebung von der Konkurrenz
        \begin{itemize}
			\item Keiner macht Zusatzleistung, höchstens Abstriche (e.g. Apotheken der Assura)
        \end{itemize}
        \item Schleichender Wettbewerb
        \item Preisdifferenzierung ist enorm wichtig -> Begründung warum Forschungsfrage interessant
    \end{itemize}
	\item Digitalisierungsvorhaben des Bundes nicht erfolgreich -> Begründung warum Forschungsfrage interessant
    \begin{itemize}
        \item TARMED, SwissMedic, PatientenDossier, …
    \end{itemize}
    \item Kunde erwartet sein Geld möglichst schnell zurück
    \begin{itemize}
        \item Automatisierte verarbeitung der Rechnung erlaubt dies  -> Begründung warum Forschungsfrage interessant
    \end{itemize}
    \item Forschungsfrage
    \begin{itemize}
        \item Abgrenzung
    \end{itemize}
\end{itemize}
	
	
\subsection{Theoretischer Teil}
Was ist Machine Learning/Deep Learning?

Supervised vs Un-supervised learning?

Welche Anwendungsbereiche gibt es aktuell?

\begin{itemize}
    \item LSTM
    \begin{itemize}
        \item Einführung in Neuronale Netzwerke
        \item Was ist ein RNN/LSTM und wie funktioniert diese Art von NN
        \item Vergleich mit CNN
        \item Faktoren für den Erfolg eines NN
        \item Training
        \begin{itemize}
            \item Merkmale geeigneter Trainingsdaten
            \item Problemstellungen
            \begin{itemize}
                \item Over-fitting
            \end{itemize}
        \end{itemize}
        \end{itemize}
		\item Anwendungsgebiete für LSTM Modelle
        \begin{itemize}
	        \item OCR
            \begin{itemize}
			    \item Pre-condition für erfolgreiches OCR
                \begin{itemize}
				    \item Gute Bildqualität
                    %\begin{itemize}
					    \item Aktuell nicht gegeben
						\item Input Kanal optimieren
                        %\begin{itemize}
							\item Verbesserung des Upload Prozess des im Kundenportal
                            %\begin{itemize}
								\item Wie viele Rechnungen werden Digital eingereicht? (aktuell ca. 2/3 digital)
                            %\end{itemize}
                        %\end{itemize}
						\item Image rectification
                        %\begin{itemize}
							 \item leptonica
							 \item OpenCV
                        %\end{itemize}
                    %\end{itemize}
					\item Eine Rechnung pro Upload
                    %\begin{itemize}
						\item Aktuell nicht gegeben
						\item TARMED einfach erkennbar
                    %\end{itemize}
                \end{itemize}				
			\item Machine translation
			\item Grammatikprüfung
			\item Image recognition
        \begin{itemize}
            \item Nicht wirklich relevant für diese Arbeit
        \end{itemize}
		\item Text Classification
        \begin{itemize}
			\item Vergleich mit klassischem NLP?
			\item Warum LSTM und nicht klassisches NLP?
        \end{itemize}
        \end{itemize}
		    \item Named Entity Recognition (NER)
        \begin{itemize}
			\item Erkennung von Namen
			\item NLP vs Machine Learning basiert
        \end{itemize}
    \end{itemize}
	\item Struktur-Anforderungen der Daten für die Weiterverarbeitung
    \begin{itemize}
		\item Syrius requirements
		\item Fachliche Anforderungen
        \begin{itemize}
    		\item KVG vs. VVG beachten
        \end{itemize}
    \end{itemize}
    \item Lösungen in anderen Branchen
    \item Lösungen von anderen Unternehmen
    \begin{itemize}
		\item Man-power
        \begin{itemize}
		    \item Assura 500 / CSS 300
            \begin{itemize}
			    \item TODO: Diese Daten sind nur "Hörensagen"
            \end{itemize}
        \end{itemize}
    \end{itemize}
\end{itemize}


\subsection{Empirischer Teil}
\begin{itemize}
    \item Methodik
    \begin{itemize}
		\item Prototyping
		\item Messkriterien des Prototypen
		\item Begrüdnung des Vorgehens
    \end{itemize}
	\item Prototyp
    \begin{itemize}
		\item OCR mittels Tesseract
        \begin{itemize}
			\item Trainieren der LSTM engine mittels artificial training data aus TARMED datenbank
			\item Artificial data wird analog eingereichter Rechnungen gemacht
            \begin{itemize}
				\item Schrfitart
				\item Noise
				\item Verzerrung
            \end{itemize}
        \end{itemize}
	    \item Spell Checking mittels eigenem LSTM Modell
        \begin{itemize}
		    \item Inspiration an DeepSpell, etc.
		    \item Training des Modells anhand artificial spelling errors basierend auf
            \begin{itemize}
				\item Google Training Datasets
				\item TARMED tarifen
				\item SYR Codes
				\item EMR Methods
				\item Other Methods
				\item Custom Tarifziffern
				\item Care Providers (Names, Addresses, …)
				\item Real Invoices
				\item Medizinischen Texten / Anamnese
            \end{itemize}
        \end{itemize}
    \end{itemize}
    \item Noise reduction
    \begin{itemize}
	    \item Pre and/or Post OCR
        \item Post OCR
        \begin{itemize}
	    	\item NN with input word + confidence -> output: noise | no-noise
        \end{itemize}
    \end{itemize}
    \item Erkennung von mehreren Rechnungen in einem Upload
    \begin{itemize}
	    \item Mehrere auf einem Bild
	    \item Mehrere auf mehreren Bildern
    \end{itemize}
    \item Klassifizierung von Rechnungen
    \begin{itemize}
		\item Traditioneller Ansatz für TARMED ja/nein
		\item LSTM Text Classification für "Prosa" (non-TARMED) Rechnungen
    \end{itemize}
    \item Strukturieren von Daten
    \begin{itemize}
		\item TARMED gut strukturierbar
		\item Für andere Rechnungen wird’s etwas schwieriger
        \begin{itemize}
			\item Patienten Erkennung
			\item Care Provider Erkennung
			\item Tabellen Erkennung
            \begin{itemize}
			    \item Erkennung der einzelnen Positionen
            \end{itemize}
        \end{itemize}
        \item Business Rules zur automatisierten Verarbeitung    
    \end{itemize}  
\end{itemize}

\subsection{Schlussfolgerungen und Empfehlungen}
\begin{itemize}
	\item Bewertung des Prototypen
	\item Beantwortung der Forschungsfrage
	\item Reflexion
	\item Empfehlungen / Zukunft
  \begin{itemize}
		\item Erweitern der Pipeline für andere Rechnungstypen (e.g. Fitnessabos)
		\item Erweitern der Business Rules
		\item On-Device OCR/NER um den upload Prozess zu verbessern
    \begin{itemize}
			\item Benutzer soll den Patientennamen nicht eingeben müssen, er soll erkannt werden
			\item Benutzer kann die erhobenen Daten direkt gegenprüfen
Bspw. Behandlungsdatum, etc.
    \end{itemize}
  \end{itemize}
\end{itemize}
 
\section{Notizen - sonstiges}
 
\subsection{Lösungsansatz ``Input Pipeline''}
\begin{itemize}
    \item Erste Optimierung bereits beim Kunden, da dort ``Crowd Sourcing'' betrieben werden kann
    \begin{itemize}
        \item Image rectification mit anpassungsmöglichkeit durch Enduser (analog OfficeLens)
        \item Quality Checking
    \end{itemize}
    \item Optimierung auf der Service Seite
    \begin{itemize}
        \item Gray Scale
    \end{itemize}
    \item OCR
    \begin{itemize}
        \item LSTM basierter Ansatz
        \begin{itemize}
            \item Vergleich zur feature detection basierten vorgehensweise (Stats, warum was besser ist)
        \end{itemize}
    \end{itemize}
\end{itemize}

Machine Learning ist sehr breit
  - Text Mining
  - Deep Learning
  - OCR
  - Classification
  - Image optimization
  
\newpage
\section{Anhang}

\subsection{Literaturverzeichnis}
\todo[inline]{Add "Abgerufen am ... von ..." to web page citations}
{
    % This renewcommand hides the title of the bibliography
    \renewcommand{\section}[2]{}%
    \bibliography{references}
}

\todo[inline]{Why the heck is there a "Literatur" heading in the toc}

\newpage
\subsection{Abbildungsverzeichnis}
{
    % This renewcommand hides the title of the listoffigures
    \renewcommand{\section}[2]{}%
    \listoffigures
}

\newpage
\subsection{Tabellenverzeichnis}
{
    % This renewcommand hides the title of the listoftables
    \renewcommand{\section}[2]{}%
    \listoftables
}

\end{document}